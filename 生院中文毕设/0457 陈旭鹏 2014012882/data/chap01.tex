\chapter{引言}
\label{cha:intro}

  
\section{液体活检与癌症早期诊断}
\paragraph{癌症早期诊断的意义和问题}
癌症又名为恶性肿瘤(Malignant tumor),指的是细胞不正常增生,且这些增生的细胞可能侵犯身体的其他部分\cite{what-is-cancer};是由控制细胞分裂增殖机制失常而引起的疾病。在人类身上,目前已知的癌症超过一百种,大多数癌症未经合理治疗都会导致死亡,其治疗难度也远大于一般疾病\cite{cancer}。癌症早期诊断患者的五年生存率要比癌症晚期患者高 5~10倍\cite{aravanis2001genetically},在晚期癌症治愈手段匮乏的情况下,癌症的早期诊断对于提高治愈率和患者的生存率至关重要。和欧美发达国家相比, 中国的癌症患者五年生存率低很多,其主要原因就是癌症早期诊断的技术不够先进,而即使在欧美国家,癌症的早期诊断技术也远没有成熟。 

\paragraph{RNA生物标志物作为液体活检指标}
 近年来,体液活检(liquid biopsy)受到人们的密切关注,体液活检相比传统的组织活检具有动态性强,无创性,成本低等特点。目前已报道的可以作为癌症检测生物标志物(biomarker)大多是蛋白质分子或者ctDNA。如2018年发表的研究CancerSEEK\cite{cohen2018detection},通过整合 ctDNA和蛋白质数据,可以从血液中对8种可能的癌症进行检出和分类, 但在该方法在确定癌症类型上的表现并不完美,准确度最低只有40\%,而检测成本可以达到500美元一个样本,成本过高。而由于RNA在中心法则中处于特殊地位,与众多的生物学过程相联系,越来越多的研究发现其在疾病发生发展中可以作为一种更有优势的标志物,RNA 标志物 与 DNA 和蛋白标志物相比,具有更好的敏感性和组织特异性\cite{xi2017rna}。 利用简单经济的一般 PCR 技术,便可以高灵敏度、高特异性地捕获和跟踪 RNA 序列。另外由于 RNA 分子在单个细胞中便拥有多个拷贝并且具有多种转录调控形态,RNA 分子标志物具备反映细胞状态与调控过程动态变化的优点。因此,大规模体液 RNA 表达数据的测定可以提供基因组差异与转录组动态变化的双重信息,可以作为准确直 接的标志物用来无创地检测人体健康和疾病状态的变化\cite{schwarzenbach2011cell},尤其是 RNA 的组织特异性克服了 ctDNA 难以从循环血中溯源的天然缺陷,对于检测和鉴定具有组织特异性的癌症具有重大的科研价值和应用前景。 

\iffalse
\paragraph{体液中的RNA(exRNA)标志物的存在形式}
随着 各种 RNA 测序方法的发展和应用,人们发现基因组中存在大量的不同于 miRNA的非编码 RNA(non coding RNA)\cite{kasowski2013extensive}。在多种体液中(如血清、唾液等)可以检测到一类非侵入性细胞外RNA(extracellular RNA, exRNA)\cite{redzic2014extracellular}。 比如环状 RNA(circular RNA)具有较稳定的空间结构,能够在血浆中稳定存在\cite{chen2016biogenesis}。这些从细胞分泌出的 exRNA 通常由胞外囊泡包裹,或者与蛋白质和脂质分子密切结合形成复合体。因为这些分子由于具备类细胞膜结构和蛋白质的保护,加上某些 RNA 具有特定的结构,exRNA 在多种体液中可以抵抗体液中 RNase的降解,从而稳定存在\cite{godoy2018large}。exRNA 的类型主要包括信使RNA(mRNA) 和多种非编码RNA:如 miRNA、piRNA、snoRNA、snRNA、tRNA、Y RNA、lncRNA、circular RNA 等。
有研究表明,外泌体 (exosomes)、微囊泡(microvesicles, MVs)、非囊泡结构的核糖核酸蛋白复合体(RNPs) 
有着不同的 RNA 组成,例如 miRNA 大量富集在外泌体中;而 RNPs 含有大量的 tRNA 和 Y-RNA 碎片,从而很容易将其与 MVs 区分开\cite{wei2017coding}。因此,在体液中具有不同的细胞外组分,且不同组分的 RNA 种类和含量很不同。 这些 exRNA 可以成为有效的生物标志物,用于癌症的早期诊断、肿瘤生长状况的动态监测、以及预后辅助诊断等过程\cite{xi2017rna}。 
\fi

\section{exRNA作为生物标志物的研究进展}

国际上已有一些规模较大的研究团队和商业组织,开始将exRNA (extra-cellular RNA)作为生物标志物进行研究。美国 NIH 下属的转化科学国家中心 (NCATS)在2013年启动了 exRNA 研究项目 ERCC(Extracellular RNA Communication Consortium)\cite{wei2017coding},研究内容包括 1) exRNA 的原理和功能; 2) exRNA作为分子标志物的可能性; 3) 基于exRNA的癌症治疗方案等。2019年的十大科技进展之一,用细胞外游离RNA(cfRNA) 信息来预测孕妇的早产风险的研究也获得了广泛的关注\cite{ngo2018noninvasive}。 在之前的体液exRNA研究中, miRNA受到了最多的研究关注,例如在一项针对肝癌的研究中,科学家使用miRNA 芯片数据区分肝癌患者和正常人群、慢性乙型肝炎患者、 肝硬化患者的血浆样本,得到 7 个分类效果最显著的 miRNA,构建了多元逻辑斯谛回归模型用以区分肝癌患者和其它对照人群\cite{zhou2011plasma}。达到了很高的敏感度(sensitivity)(约80\%-90\%)和特异性 (specificity)(约70\%-80\%),不过依然有20\%左右的误诊率提升空间。


\paragraph{exRNA生物信息学分析方法的发展与挑战}
为了解析体液 exRNA-seq数据,必须针对其特征设计专门的生物信息学工具。针对 exRNA-seq数据的非均一化、易降解、碎片化、杂音大、具有批次效应、动态性更强等特点,目前尚缺乏专业和完整的生物信息学分析方法。例如,不同批次(如不同实验日期,不同实验条件等)取得体液样本之间存在很大差异。实验条件的不同也导致不同样本的库大小并不一致,因此还需要进行样本库大小的归一化。 除此以外,RNA 分子除了能够反映基因组变异的信息,同时后转录调控过程使得 RNA 分子具有广泛的多态性,血液样本中的 RNA 分子不同于或组织中RNA的存在形式,往往受到降解作用的影响,多以碎片的形式存在,传统的构建表达矩阵(基因计数矩阵)的方法不够精确。同时由于exRNA的高度动态性和微量性,对于挑选出稳健的生物标志物也提出了很大的挑战。

现在已经有几种用于小RNA-seq分析的工具:ExceRpt\cite{kaczor2017novel},用于细胞外sRNA的整合基因组分析的工具(TIGER) \cite{allen2018bioinformatic},TIGER的一个关键改进是能够分析亲本RNA和单个片段水平以及宿主和非宿主sRNA。Chimira\cite{vitsios2015chimira}是一种基于网络的系统,用于小RNA-seq数据中miRNA的分析。包含自动清理,修剪,大小选择并直接比对到miRNA发夹序列等。产生表达矩阵用于随后的统计分析。Chimira还提供了一套简单直观的工具,用于分析和解释比对结果。miRge \cite{baras2015mirge}是一种处理小RNA-Seq数据以获得microRNA熵的方法。miRge使用贝叶斯比对方法,按照顺序与成熟的miRNA,发夹miRNA,非编码对齐
RNA和mRNA比对。其他种类RNA(tRNA,rRNA,snoRNA,mRNA)也可以比对。 miRge能够在52分钟内同时分析100个小RNA-seq样品,提供关于miRNA表达的综合分析。Oasis\cite{capece2015oasis}是一个Web应用程序,可以快速灵活地在线分析小RNA-seq数据。它是专门针对实验室终端用户设计的工具,提供易于使用的Web前端和演示视频教程,演示如何一步步地分析sRNA-seq数据。 Oasis包含差异表达模块以及用于稳健生物标志物检测的模块以及GO和通路富集分析。支持批量提交任务。 Oasis可以生成可下载的交互式Web报告,以便在本地系统上轻松可视化和分析数据。 一些应用于RNA-seq和single cell数据的标准化和批次效应校正工具(包括scImpute\cite{li2018accurate},SCnorm\cite{bacher2017scnorm}和Combat\cite{chen2011removing}
)也可能用于exRNA数据的分析。 但是针对exRNA测序数据的一些关键问题,如高噪音、碎片化,特征不够稳定等问题,目前还没有一个通用的整合性的工具可以兼顾到这些问题,完成exRNA数据的分析以及潜在生物标志物的发掘。



\section{机器学习与特征选择算法}
\label{sec:third}

机器学习可以利用计算机、数学、统计等方法,利用计算机深入挖掘数据的内部分布、潜在特征等,完成对数据模式的学习和识别。机器学习(machine learning, ML)的方法可以从复杂的数据分布中发现数据的相关特征,提取出数据中最重要的特征,并且具有很强的数据拟合能力,一些经典的机器学习模型,如支持向量机(support Vector Machines, SVM),逻辑斯谛回归(Logistic Regression,LR),随机森林(Random Forest, RF),决策树(decision trees, DTs)等,已经被广泛应用于生物学数据分析中,包括疾病尤其是癌症的分类和检测中。


常见的机器学习方法包括有监督学习和无监督学习。有监督学习中,模型通过获得输入数据以及对应的每个样本的标签,来学习输入数据和输出标签的内在映射,有监督学习包括分类和回归两类,对于标签为离散值的预测问题(如癌症,正常人两类标签)被称为分类问题,标签为连续值的问题为回归问题。与有监督学习不同,无监督学习不提供输入标签,因此模型需要将输入的样本进行聚类,每个类别具有类似的特征。

有监督学习的分类模型要求输入数据集和对应的类别标签对模型进行训练,之后模型可以对新输入的数据进行预测。数据集一般是数值矩阵,如本问题中使用的exRNA-seq测序数据构建出的基因表达矩阵,除此之外,基因组,蛋白质组,代谢组,图像等数据都可以作为数据集输入。类别标签为癌症和正常人,可以被数值化以便输入模型。在训练模型并进行模型预测和模型评估时,还需要进行交叉验证(cross validation, CV),交叉验证通过将数据分为训练集和测试集,在训练集上训练数据,在独立的测试集上测试数据,可以很好地反应模型的泛化能力。最常见的 CV 方法为K折交叉验证(K-fold CV),在 K-fold CV 中,数据集被划分为 K 个互斥的子集,分类器每次在 K-1 个子集上训练并在 1 个子集上进行测试,直到每个子集都被用作测试数据集。 最终通过计算K折上的平均准确度作为模型的准确度。


在机器学习和统计学中,特征选择(feature selection)也被称为变量选择、属性选择 。即为了构建模型而选择相关特征(即属性、指标)子集的过程\cite{featureselection}。使用特征选择技术可以简化模型,增加模型解释性,缩短训练时间,降低过拟合等。如果训练数据包含许多冗余或无关的特征,就可以移除这些特征而不损失信息。对于癌症检测以及挑选生物标志物的任务而言,挑选出少量的生物标志物的同时保证较高的预测准确率、灵敏度和特异性,对于实际应用非常有意义。通过选择不同的评价指标,可以把特征选择算法分为三类:包装类(wrapper)、过滤类(filter)和嵌入类(embedding)方法。嵌入式方法通过在分类算法中构建特征选择来评估最佳特征子集。与过滤方法相比,嵌入式方法更加节省计算资源。特征选择方法对于决定最终的分类效果非常重要,由于基因组学数据往往具有很高的特征维度和很低的样本维度,因此选择少量的足够有代表性的特征非常重要,选择稳定的,有解释性的特征具有相当的挑战性。


不同的 ML 技术和特征选择算法已被广泛应用于癌症的预测和预后。如使用多元逻辑斯谛回归区分肝癌和正常样本的研究\cite{zhou2011plasma},使用带有正则化的随机森林方法进行不影响预测性能的特征选择用于癌症检测\cite{liu2014learning}等。



\section{研究计划概述}
\label{sec:fifth}

exSEEK的主要功能如图~\ref{fig:pipeline}所示,包括测序数据的清洗,质量控制,序列比对映射,对于小RNA数据我们专门探索了顺序比对的方法。对于构建表达矩阵(基因计数矩阵),我们针对细胞外RNA测序数据的特点,设计了结构域检测的算法,检测lncRNA、mRNA、snoRNA、snRNA、srpRNA、tRNA、Y RNA等RNA的结构域特征用于构建表达矩阵,并与miRNA合并产生计数矩阵,之后我们使用一系列的矩阵处理模型对表达矩阵进行测序深度的归一化以及去除批次效应,并且使用针对性的指标衡量矩阵处理的效果。最后我们会构建特征选择的流程来挑选可以用于癌症分类的潜在生物标志物,使用一些机器学习模型进行特征的挑选,并对其分类效果进行评估。
最后我们会将代码封装为一个操作简单的命令行软件,并且提供可交互的可视化模块方便终端用户进行数据的可视化和分析。

本课题是是Lu lab的一项研究的一部分,我负责了测序数据的处理,表达矩阵的处理以及简单的机器学习模型的搭建和可视化分析部分,完成了所有代码中的约40\%的工作。
\begin{figure}[H] % use float package if you want it here
    \centering
    \includegraphics[width = 0.8\textwidth]{pipeline}
    \caption{exSEEK流程图}
    \label{fig:pipeline}
\end{figure}

\section{研究意义}


我们已经总结了体液检测对癌症早期诊断和预后的意义,其中使用exRNA作为分析数据鉴定癌症尤其是早期癌症是非常有潜力和意义的研究。目前针对exRNA-seq数据的分析工具和方法依然缺乏,大多数都来自于其他类型数据(如RNA-seq,single cell)分析工具的迁移。这些方法依然无法解决exRNA-seq数据中的一些关键问题,如数据的高度碎片化和噪音,数据的微量性和不稳定性。同时也欠缺一个覆盖所有流程的统一的exRNA-seq分析工具。我们通过进行精细的序列比对,质量控制,结构域检测,样本库归一化,去除批次效应,以及进行稳定特征的选择等步骤,构建了一个操作简单的exRNA-seq分析和潜在生物标志物挑选工具,可以作为exRNA-seq数据分析的有力工具。相信在未来随着exRNA-seq测序数据的大量产生,机器学习模型将在矩阵处理和特征选择中发挥更加重要的作用,可以对exRNA-seq数据的分析和生物标志物挖掘发挥更加重要的作用。





%~\ref{fig:data_issues}
%\begin{figure}[H] % use float package if you want it here
%    \centering
%    \includegraphics[width = 0.95\textwidth]{data_issues}
%    \caption{exRNA-seq数据分析中的挑战}
%    \label{fig:data_issues}
%\end{figure}