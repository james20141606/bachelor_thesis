\chapter{总结与讨论}
\label{cha:conclusion}

\section{结论}

在本研究中,我们完成了exRNA-seq的完整的生物信息学分析流程,包括L-exoRNA-seq和S-cfRNA-seq,S-exoRNA-seq数据的比对(小RNA-seq数据采用顺序比对),构建基因表达矩阵(对于小RNA-seq数据采用结构域检测的方法获得结构域特征),使用一系列的矩阵处理方法完成样本库的归一化和批次效应的去除,并且使用UCA和mKNN两个分数衡量处理效果,最后使用一个特征选择框架完成特征的挑选以及在多套数据和癌症类型上进行分类效果的评估以及生物标志物的分析,我们挑选出了一些非编码RNA和miRNA的组合,可以取得比已知miRNA生物标志物更好的分类效果。最后我们将相关代码封装,制作成一个易于使用的工具exSEEK,方便终端用户使用exSEEK完成exRNA-seq的全流程分析,并且提供可交互的,可以方便可视化的工具供用户分析和理解处理结果。

\section{讨论}

\paragraph{exRNA作为生物标志物的生物学原理}

细胞外RNA(exRNA)可能富集于外泌体 (exosomes)、微囊泡(microvesicles, MVs)、非囊泡结构的核糖核酸蛋白复合体(RNPs) 中,与组织细胞中的RNA差异较大,细胞外RNA已经被证明可能与癌症细胞相关,作用于癌症的发生,转移等过程,并且影响免疫系统。exRNA由于被分泌到细胞外,且能够稳定存在的exRNA往往具有较为稳定的RNA二级结构或者与RNA结合蛋白一起形成复合物。这种生物学上的选择在原理上证明了exRNA作为生物标志物的潜力,通过我们的分析结果,可以发现exRNA数据在癌症的鉴定,尤其是早期癌症的检测上,具有较好的准确率,灵敏性以及特异性。


\paragraph{矩阵处理的改进}

exRNA数据存在明显的异质性和批次效应,目前我们使用一些RNA-seq和单细胞领域的处理方法,这些方法往往基于一些统计学的假设建模,并没有充分考虑到exRNA数据的特征,因此进行矩阵处理的方法还有待改进。比如对表达矩阵进行样本库大小的归一化,由于exRNA数据的稀疏性和少数基因占据主导的特点,归一化方法可能不够稳定,即使使用外部加入的含量固定的spikein RNA,也有可能因为实验技术的问题导致不同样本的spikein并不完全一致。另外我们还遇到了批次信息和样本类别信息完全重合的情况,如何对这样的数据进行批次效应的处理,以及更好地衡量批次效应的去除还值得探索,2018年的一篇文章探讨了衡量批次效应去除的指标kBET(\cite{buttner2019test}),我们也提出了mKNN分数来衡量批次效应去除效果,比kBET在小样本上更加稳定,但是一个公认的方便使用的批次效应衡量指标依然有待研究。



\paragraph{机器学习模型的小样本和过拟合}

机器学习问题中,为了避免过拟合,测定模型的泛化性能,一般使用交叉验证的方式。在特征选择部分,针对exRNA-seq样本量少,特征维度高,容易过拟合的问题,我们使用了50轮随机抽样进行交叉验证,测试模型的泛化能力。另外,研究中使用的模型也都具有一定的避免过拟合的能力,如随机森林模型在处理测序数据时具有一定的避免过拟合的能力(~\cite{diaz2006gene}),而逻辑斯谛回归也可以通过加入$L_1$正则化等方式对特征挑选进行约束,一定程度上进行正则化,避免过拟合(\cite{ng2004feature})。在未来获得更大量样本的情况下,过拟合的问题可能会得到一定程度的缓解,机器学习模型的结果将会更加可靠。伴随小样本问题的另一问题是同一套数据内类别不均衡的问题(~\cite{chawla2004special}),常规的处理方法一般采取对样本量少的类别升采样,或者对样本量多的类别降采样,由于本身数据量很小,降采样的方式并不可取,然而升采样时也必须注意根据数据的分布进行采样,复杂的升采样方法常采用生成模型,如自编码器(VAE),生成对抗网络(GAN)等,但是对于样本量本身比较少的数据也并不容易实现。因此如何均衡数据的类别也需要更仔细的思考。




机器学习的另一个限制是生成的模型通常不易解释。即使机器学习算法非常有 效,我们通常也无法理解其中的算法结构与基础生物学之间的对应关系。通过机器 学习算法得到的具有区分不同表型的重生物标志物不一定与疾病的生物学 发生或 发展有着显着的直接关系。例如,生物标志物可以由感兴趣的疾病过程下游的免疫 应答产生。由于我们难以破译驱动机器学习算法的具体机制,因此在尝试将算法应 用于与训练对象非常不同的新群组时应更加谨慎,例如从动物模型转换为人类临 床样本。



\paragraph{模型的可解释性以及与其他数据结合的可能性}
机器学习模型的一个问题和局限是其本身的黑箱特性,即内部的原理难以解释,不容易与实际问题的原理对应。我们挑选出的特征可能难以直接与癌症的生物学机理对应,挑选出的特征的组合也难以被诠释出具体的意义。为了解决这个问题,研究人员从不同角度对模型进行修改,如使用变分自编码器的中间层隐含变量对生物学机制进行诠释,构建基于图的模型对调控网络进行建模等,以增强模型的可解释性。另一方面,研究人员也在探索组合使用多种类型数据进行癌症的预测,如结合甲基化,RNA编辑和剪接数据以及图像数据等(\cite{cappelli2018combining},\cite{cheng2017integrative},\cite{wang2018unifying}),对于不同数据的组合,一般可以在建模的不同阶段将其融合,如在数据准备阶段将不同的数据拼接成一个高维张量(tensor),或者对不同类型的数据建立不同的模型,在预测阶段将模型融合等。利用更多来源的数据可以互相补充,加入更多的信息,有助于提高预测的准确率和灵敏性,特异性等。


总之我们认为,在未来随着exRNA-seq测序技术的不断发展和样本量的积累,以及exRNA-seq分析技术以及特征选择算法的发展,人们将可以挑选出表现更好,泛化能力更强,稳定性更好而且具有更强解释力的基因作为生物标志物,并且广泛应用于癌症尤其是早期癌症的检测与预后治疗中。
