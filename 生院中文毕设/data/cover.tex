\thusetup{
  %******************************
  % 注意:
  %   1. 配置里面不要出现空行
  %   2. 不需要的配置信息可以删除
  %******************************
  %
  %=====
  % 秘级
  %=====
  secretlevel={秘密},
  secretyear={10},
  %
  %=========
  % 中文信息
  %=========
  ctitle={基于细胞外RNA测序数据的生物标志物鉴定工具开发},
  cdegree={理学学士},
  cdepartment={生命科学学院},
  cmajor={生命科学},
  cauthor={陈旭鹏},
  csupervisor={鲁志\ 教授},
 % cassosupervisor={陈文光教授}, % 副指导老师
 %ccosupervisor={某某某教授}, % 联合指导老师
  % 日期自动使用当前时间,若需指定按如下方式修改:
  % cdate={超新星纪元},
  %
  % 博士后专有部分
  cfirstdiscipline={计算机科学与技术},
  cseconddiscipline={系统结构},
  postdoctordate={2009年7月——2011年7月},
  id={编号}, % 可以留空: id={},
  udc={UDC}, % 可以留空
  catalognumber={分类号}, % 可以留空
  %
  %=========
  % 英文信息
  %=========
  etitle={An Introduction to \LaTeX{} Thesis Template of Tsinghua University v\version},
  % 这块比较复杂,需要分情况讨论:
  % 1. 学术型硕士
  %    edegree:必须为Master of Arts或Master of Science(注意大小写)
  %             “哲学、文学、历史学、法学、教育学、艺术学门类,公共管理学科
  %              填写Master of Arts,其它填写Master of Science”
  %    emajor:“获得一级学科授权的学科填写一级学科名称,其它填写二级学科名称”
  % 2. 专业型硕士
  %    edegree:“填写专业学位英文名称全称”
  %    emajor:“工程硕士填写工程领域,其它专业学位不填写此项”
  % 3. 学术型博士
  %    edegree:Doctor of Philosophy(注意大小写)
  %    emajor:“获得一级学科授权的学科填写一级学科名称,其它填写二级学科名称”
  % 4. 专业型博士
  %    edegree:“填写专业学位英文名称全称”
  %    emajor:不填写此项
  edegree={Doctor of Engineering},
  emajor={Computer Science and Technology},
  eauthor={Xue Ruini},
  esupervisor={Professor Zheng Weimin},
  eassosupervisor={Chen Wenguang},
  % 日期自动生成,若需指定按如下方式修改:
  % edate={December, 2005}
  %
  % 关键词用“英文逗号”分割
  %ckeywords={\TeX, \LaTeX, CJK, 模板, 论文},
  %ekeywords={\TeX, \LaTeX, CJK, template, thesis}
}

% 定义中英文摘要和关键字
\begin{cabstract}
  体液中的细胞外RNA(exRNA)可以为癌症鉴定提供了大量候选生物标记物,
  从而实现癌症尤其是早期癌症检测。深度测序的发展使得全面地检测exRNA成为可能,然而由于exRNA测序数据的独特性质,从生物信息学的角度分析exRNA-seq并鉴定可临床使用的生物标志物仍然是一项非常有挑战性的工作。我们为此开发了exSEEK, 作为一种用于exRNA-seq分析和鉴定与癌症相关的生物标志物的生物信息学工具,可用于不同来源(细胞外游离、外泌体分泌)以及不同测序方法(小RNA测序、长RNA测序)产生的exRNA数据分析及生物标志物鉴定。
  
  exRNA测序数据具有高度碎片化、稀疏性、异质性和批次效应等特征。我们通过对exRNA-seq进行精细的比对,构建小RNA-seq基因表达矩阵时使用“结构域特征”,对表达矩阵进行多种样本库大小归一化和去除批次效应方法进行组合,使用非监督聚类准确性(UCA)和m-K最近邻(mKNN)指标衡量其对计数矩阵的归一化和去除批次效应效果。我们还开发了一个特征选择框架,使用逻辑斯谛回归,随机森林等机器
  学习模型用于选择区分癌症和正常样本的有效且稳定的特征。我们使用
  exSEEK对三个数据集进行了综合分析并评估了exSEEK在肝癌,结肠癌,胰腺癌
  和前列腺癌等癌症的分类性能,取得了良好的效果。最后我们还将exSEEK封装成一个简单易用的软件,并且提供交互性的可视化模块供用户使用。
  
  
  \end{cabstract}
  
  \ckeywords{液体活检, 生物标志物, 特征选择, 机器学习}
  
  \begin{eabstract}
  Extracellular RNAs (exRNAs) in body fluid provide a large 
  repository of biomarker candidates for cancer (early) diagnosis. Deep sequencing makes it possible to monitor exRNAs in a comprehensive way. However, due to the unique properties of exRNA sequencing data,  analyzing exRNA-seq data and identifying potential biomarkers for clinical usage 
  remains challenging. Here we developed exSEEK, as a bioinformatics tool for 
  identification of biomarkers associated with certain 
  diseases by analyzing both small 
  and long exRNA sequencing data generated by both cell 
  free or exosome RNA.
  
  
  exRNA-seq data are difficult to deal with since it is highly fragmented, sparse heterogeneous and has batch effect. In this work, we did careful mapping, constructing expression matrix using domain features.  We 
  applied various combinations of normalization and 
  batch removal methods to the count matrix to correct 
  data heterogeneity and batch effects. We evaluate 
  the normalization and batch correction result using 
  unsupervised clustering accuracy (UCA) and m-K-nearest 
  neighbor (mKNN). We also developed a feature selection 
  framework wrapping some machine learning models 
  including logistic regression and random forest to 
  robustly select the most important features that distinguish 
  cancer from normal samples. We performed integrative 
  analysis of three datasets: cell-free small RNA, 
  exosomal small RNA and exosomal long RNA and evaluated 
  the classification performance of HCC, CRC, PRAD, and PAAD. At last we wrapped all the useful functions into a easy-to-use software with interactive visualization modules for end users.
  \end{eabstract}
  
  \ekeywords{liquid biopsy, biomarker, feature selection, machine learning}