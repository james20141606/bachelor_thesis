\chapter{Literature Review}
\label{cha:engorg}

\title{Literature review for the project "Developing bioinformatics methods on exRNA-seq analysis for cancer"}

\section{Research significance and scientific basis of the project}

Compared to DNA and protein biomarkers, RNA has higher tissue specificity, sensitivity, and lower testing costs.
These advantages of RNA have made it promising to serve as an alternative, or even more advantageous biomarker in cancer diagnosis$^{[1],[2]}$. An RNA molecule that can be secreted extracellularly is called exRNA (extracellular 
RNA) and is a major component of RNA that can be detected in body fluids. Currently, large-scale international
research teams (such as the Extracellular RNA Communication Consortium)$^{[3]}$ and commercial organizations 
(such as the Gates Foundation, etc to invest 2 billion company GRAIL) has exRNA as the object of focus 
on exploration, it can be used as desired biomarkers in body fluid biopsies. Although exRNA contains a 
variety of RNA types, such as mRNA fragments and a variety of non-coding RNAs such as miRNA, piRNA, snRNA, 
lncRNA and circular RNA, the mainstream research of exRNA at this stage is mostly concentrated on star molecules such as miRNA. More than that, these RNAs have unusually complex processing engineering, 
such as alternative splicing, editing, etc., which generate a large number of RNA variants (isoforms) 
in body fluids, increasing the difficulty of these exRNA studies. On the other hand, if we can use the diversity of messenger RNA, we most likely to break through this stage of clinical testing means, 
promote the development of early screening for cancer$^{[1]}$. The extraction of these complex, 
high-throughput diversity information relies heavily on the development of specialized bioinformatics
methods and the development of microsequencing technologies.

To be specific, liquid biopsies comprising the noninvasive analysis of circulating tumor-derived material (the "tumor circulome"), represent an
innovative tool in precision oncology to overcome current limitations associated with tissue biopsies. The "tumor circulome", defined as the subset of circulating components, is derived from cancer tissue and can be directly or indirectly used as a source of cancer biomarkers in liquid biopsies$^{[4]}$. These components include circulating tumor proteins, 
circulating tumor nucleic acids (ctDNA and ctRNA), CTCs, Extracellular vesicles (EVs), and tumor-educated platelets (TEPs). EVs, ctRNA, and 
TEPs are relatively new tumor circulome constituents with promising potential at each stage of cancer management. EVs are membranous particles released from all cell types under physiological and pathological conditions, as well as following different types of stimuli, including proteases, ADP, thrombin, inflammatory cytokines, 
growth factors, biomechanical shear, and stress inducers, and apoptotic signals $^{[5]}$. They can be found in almost every bodily fluid, especially blood $^{[6]}$. EVs have been recognized as fundamental mediators of intercellular communication, regulating and participating in a 
plethora of physiological and pathological processes, including cancer $^{[6]}$. Based on their biogenesis, content, and secretory 
pathways, EVs can be divided into two broad categories: 
exosomes and microvesicles$^{[6]}$.

The suitability of EVs as cancer biomarkers lies in the fact that the molecular cargoes they carry can be considered a molecular fingerprint of the cell of origin$^{[7]}$. Similar to ctDNA and CTCs, EVs can be a source of quantitative and qualitative information. Quantitative information comprising EV numbers can inform the presence of malignant disease and tumor burden. For example, circulating exosome levels are 
increased in breast and pancreatic cancer$^{[8]}$ and the number of 
circulating microparticles (MPs) is higher in patients with multiple 
myeloma (MM) compared with healthy individuals$^{[9]}$.

\paragraph{RNA markers are characterized by diverse species and complex 
forms, and have important scientific research value} 
With the development and application of various RNA sequencing methods, 
it has been found that there are a large number of novel non-coding 
RNAs$^{[10]}$ different from miRNAs in the genome. In the past, 
research on the use of RNA sequencing for disease diagnosis mainly 
focused on changes in transcript expression levels caused by certain 
diseases$^{[11]}$, gene fusion$^{[2]}$ and so on, but with the 
development and integration of RNA high-throughput sequencing 
technology and bioinformatics technology, people found much more 
information than that, and detected clinically relevant RNA species$^{[1]}$ complex, 
comprising: different types of RNA (mRNA, miRNA, tRNA, Y RNA, snoRNA, 
circRNA, lncRNA, etc.), different isoforms, modified species, 
intracellular and extracellular distribution, etc. 
The discovery and understanding of RNA diversity, 
has a very important scientific value; at the same time, 
the use of RNA biodiversity information is also likely 
to break at this stage of clinical testing methods to 
promote early cancer screening.

The detection of clinically relevant RNA species is complex, 
including different RNA types (mRNA, tRNA, miRNA, Y-RNA, snoRNA, 
circRNA, lncRNA, etc.), different splicing forms, modified species, 
intracellular and extracellular distribution, etc. The RNA content of EVs, including 
both coding and noncoding (nc)RNAs, has been widely studied≈.




\section{Current research status internationally}
The research progress related to the RNA marker research of this subject
is briefly introduced as follows:

\paragraph{Internationally, some large-scale research teams and commercial organizations have begun 
to conduct exploratory research on exRNA (extra-cellular RNA) as a biomarker.}
Recently, the National Center for Translational Science (NCATS) under the NIH of the United 
States launched the exRNA research project ERCC (Extracellular RNA Communication
Consortium) 3, which includes several research directions, including 1) exRNA 
treatment protocols; 2) molecular markers; 3) The mechanism of action and 
other aspects have also funded multiple research groups to conduct research. 
Mainly to detect EML4-ALK gene fusion and EGFR T790M gene mutation in plasma samples$^{[14]}$. 
The first representative product is ExoDx lung (ALK), EML4-ALK is used to detect non-small cell lung cancer patient's plasma exosomes transcripts. In addition, 
many teams began using RNA circulating in the mother's body fluids of information 
to reflect the health of the fetus.

\paragraph{In previous studies of body fluid exRNA, miRNA was a star molecule and received extensive attention.} miRNAs can be endogenously expressed in a variety of cells and secreted into a variety of body fluids (blood, saliva, and urine). 
Based on these features, miRNA can be used as a non-invasive biomarker to become 
one of the ideal candidate biomarkers of human diseases including cancer$^{[16]}$.A recent 
study of non-small cell lung cancer (NSCLC, accounting for 85\% of all cases of lung 
cancer) in patients with early plasma samples were genome-wide miRNA expression 
profiling, miRNA found 24 kinds of circulation, has a high diagnostic value$^{[17]}$; 
another study, serum of miRNA expression has undergone radical prostatectomy for 
prostate cancer patients were analyzed, the results found 43 kinds of miRNA, can 
distinguish between different disease stages 14 prostate cell lines and patient 
samples$^{[18]}$. China has also made significant achievements in the research and 
application of miRNA biomarkers. For example, in a study of liver cancer, 
Chinese scientists using miRNA microarray patients with hepatocellular carcinoma 
and the normal population, patients with chronic hepatitis, liver cirrhosis 
patient plasma samples, and screened seven most significant effect of miRNA, 
the establishment of a multi-index Logistic regression analysis model to 
distinguish between liver cancer patients and other control populations$^{[19]}$. 
The kit based on this result has completed multi-center clinical validation and recently passed the certification of the State Food and Drug Administration. 
Many of these detection methods based on miRNA has reached a very high (about 80\% -90\%)
sensitivity (sensitivity), and specificity (specificity) is also very good, up to 
about 70-80\% and more; but it also shows There are still about $\frac{1}{4} \text{to} \frac{1}{5}$
misdiagnosis rates and room for improvement.

\paragraph{The discoveries and research of new exRNA progress quickly, exRNA types that can be used as biomarkers are far more than miRNA. } Nearly all known classes of RNA have been found in systemic circulation and, 
to a certain extent, each has the potential to serve as a cancer biomarker$^{[20]}$. 
The most important classes of ctRNA potentially suitable as biomarkers are mRNAs, miRNAs, and long ncRNAs (lncRNAs). Their analysis is performed with techniques ranging from qRT-PCR or dPCR-based assessment of single or small panels of RNAs to the comprehensive characterization of RNA (especially miRNAs) signatures via RNASeq20. By definition, a variety of extracellular RNA are collectively referred to as exRNA, they play an important role in the communication between cells, can be transported to tissues adjacent to or distant, it is taken up and transferred they carry genetic regulation of target cells and information. 
exRNA would normally be wrapped into the exosomes (with exosomes), microvesicles vesicles (microvesicles, MVs)$^{[21]}$ and the like, RNA and protein complexes of non-vesicular structures (RNPs)$^{[22]}$, which have different grain. The size of the diameter can usually be separated by ultracentrifugation and physical sedimentation $^{[23]}$ and can be distinguished according to different surface markers. 
Studies have shown that MVs, exosomes, and RNPs have different RNA compositions, 
such as miRNAs that are abundantly enriched in exosomes; RNPs contain large amounts of tRNA and Y-RNA fragments, making it easy to extract them from MVs$^{[24]}$. 

Circulating exosomal mRNA has been used to investigate the mutational status of KRAS 
and BRAF in patients with CRC$^{[25]}$, and exosomal EGFR vIII mRNA has the potential for the 
diagnosis of EGFRvIII-positive high-grade gliomas$^{[26]}$. In another report, the 
detection of androgen receptor splice variant 7 (AR-V7) in plasmatic exosomes 
by ddPCR was shown to be a good predictor of resistance to hormonal therapy 
n prostate cancer$^{[27]}$. Numerous lung cancer-related gene fusions are also 
readily identified in both vesicular and nonvesicular mRNA and have value as 
biomarkers$^{[27]}$. Among the nonvesicular fraction of ctRNAs, circulating human 
telomerase reverse transcriptase (hTERT, the catalytic subunit of the telomerase 
complex) mRNA demonstrated greater diagnostic and prognostic accuracy than 
PSA for prostate cance$^{[28]}$.

With regards to miRNAs, plasma exosomal miR-196a and miR-1246 levels have the potential 
for the early diagnosis of pancreatic cancers$^{[29]}$, and panels of miRNAs have been shown 
to be reliable biomarkers for the diagnosis$^{[30]}$ or prognosis31 of lung cancer. More 
recently, a serum exosomal miRNA signature was proven to be an innovative tool for
the differential diagnosis of gliomas$^{[31]}$.

It is also well-known that there is a lncRNA called PCA3 and it has been identified as a molecular marker in the urine for prostate cancer. For example, plasma exosome 
LINC00152 levels have been linked to gastric cancer$^{[32]}$, and the combination of two 
mRNAs and one lncRNA in serum exosomes have diagnostic potential for CRC$^{[33]}$. 
Furthermore, serum exosomal HOTAIR lncRNA has applicability in the diagnosis and prognosis of glioblastoma multiforme$^{[34]}$. More recently, a panel of five circulating lncRNAs was studied as promising diagnostic biomarkers for gastric cancer $^{[35]}$. 

A recent study shows that circular RNAs can be detected in urine and has the potential to serve as prostate cancer's biomarker$^{[36]}$. Circular RNAs (circRNAs) are single-stranded, 
covalently closed RNA molecules that are produced from pre-mRNAs through a process called back splicing and were initially proposed to be splicing-associated noise$^{[37]}$. 
Recent studies have shown that circRNAs may be involved in microRNA (miRNA) inhibition$^{[38]}$, 
epithelial-mesenchymal transition$^{[39]}$, and tumorigenesis$^{[40]}$. Further, circRNA expression can be tissue specific$^{[39]}$, and some evidence supports the translation of some circRNAs$^{[41],[42]}$.
CircRNAs are highly stable and can be found in exosomes, cell-free saliva, and plasma. 
Therefore, with improved detection and characterization methodologies, circRNAs may be 
potential biomarkers or therapeutic targets.

Therefore, there are different extracellular components in the body fluid, and the RNA 
species and content of the different components are very different. 

\paragraph{In order to sequence exRNAs in body fluids, high-throughput sequencing technology for 
low-input RNAs is one of the key issues.} The amount of free RNA in body fluids is low$^{[43]}$, 
and RNA itself is easily degraded by endogenous and exogenous RNase, so some specialized 
methods and commercial kits have been established for the extraction and sequencing of 
trace RNA, for example, in single cell transcription. and genomic experiments, building
an RNA library can be a single cell sequencing of RNA picograms (pg) level, the main 
application is SMARTer$^{[44]}$ library construction sequencing technology and other technical
MALBAC$^{[45]}$, these two techniques were used to capture trace the template switch full-length 
RNA and the use of multiple annealing circular loop amplification techniques allow the
ends of the amplicon to complement each other to prevent exponential amplification.

In recent years, a variety of different RNA library construction methods have been developed for exRNA sequencing, and the results obtained by different RNA library construction methods have significantly different results. The cell-free RNA-seq obtained
by four different RNA-seq methods was significantly different in rRNA ratio, 
library size, microbiome, and detectable different types of RNA. 

\paragraph{In addition to the abundance of RNA expression, variants and isoforms resulting from 
post-transcriptional regulation of RNA can also serve as markers for the development 
of cancer.} Studies have found that in cancer cells, compared with normal cells, 
RNA splicing, poly (A) tailing and editing and other post-transcriptional regulatory events are misregulated. Shen et al. systematically identified a number of alternative
splicing events$^{[46]}$ associated with the clinical outcome of cancer by analyzing TCGA data. 
Xia et al. also analyzed the data of TCGA and found that in cancer cells, a large number of cancer-related genes are regulated by alternative polyadenylation, and the 3'UTR 
shortened genes are prone to be up-regulated, demonstrating alternative polyadenylation is also closely related to the development of cancer$^{[47]}$. In recent years, more and more studies have found that RNA editing events are also erroneously regulated in cancer patients. Many RNA editing sites in mRNA and miRNA are identified due to differences in editing levels between cancer patients and normal people. For potential 
biomarkers$^{[48],[49]}$.




\section{Previous analyzing tools for small RNA-seq}
There are several existing tools for small RNA-seq analysis: ExceRpt$^{[50]}$, 
TIGER$^{[51]}$, Chimira$^{[52]}$, miRge$^{[53]}$, and Oasis$^{[54]}$. Some imputation, normalization
and batch correction tools including scImpute, SCnorm, and combat 
may also be useful.

The exRNA-seq dataset has certain properties: fragmented, sparse, 
and heterogeneity. Some tools consider its fragmented characteristics and there are many scRNA-seq analyzing tools for normalization and batch correction. 

Tools for Integrative Genome analysis of Extracellular sRNAs (TIGER) 
was performed on mouse lipoproteins, bile, urine, and livers. A key advance for the TIGER pipeline is the ability to analyze both host and non-host sRNAs at genomic, parent RNA and individual fragment levels. Moreover, TIGER facilitated the comparison of lipoprotein 
sRNA signatures to disparate sample types at each level using hierarchical clustering, correlations, beta-dispersions, principal coordinate analysis and permutational multivariate analysis of variance. TIGER analysis was also used to quantify distinct features 
of exRNAs, including 5ʹ miRNA variants, 3ʹ miRNA non-templated 
additions and parent RNA positional coverage. The researchers have observed that miRNA explained less than 5\% of quality sequencing depth of lipoproteins and only ~15\% of liver sequencing depth. 
It seems that non-coding RNAs are processed to smaller fragments creating an enormously diverse pool of sRNAs in cells and extracellular fluids. They found there are several sRNAs are derived from tRNA, snRNA, and rRNA. TIGER also have more advanced 
ability to analyze sRNAs since other tools are restricted to 
miRNAs or endogenous (host) sRNAs, including Chimira, Oasis, and miRge.

Chimira is a web-based system for microRNA (miRNA) analysis from small RNA-Seq data. Sequences are automatically cleaned, trimmed, 
size selected and mapped directly to miRNA hairpin sequences. This generates count-based miRNA expression data for subsequent statistical analysis. Moreover, it is capable of identifying epi-transcriptomic modifications in the input sequences. Supported
modification types include multiple types of 3'-modifications 
(e.g. uridylation, adenylation), 5'-modifications and also internal
modifications or variation (ADAR editing or single nucleotide polymorphisms). Besides cleaning and mapping of input sequences to miRNAs, Chimira provides a simple and intuitive set of tools for the analysis and interpretation of the results. Chimira is a 
a fast and robust system for the cleaning, filtering, QC and mapping 
of small RNA-Seq data aiming to simplify the process of small RNA 
NGS analysis to a straightforward online workflow.

miRge is a Multiplexed Method of Processing Small RNA-Seq Data to 
Determine MicroRNA Entropy. It is a fast, smart small RNA-seq solution to process samples in a highly multiplexed fashion. miRge employs
a Bayesian alignment approach, whereby reads are sequentially 
aligned against customized mature miRNA, hairpin miRNA, noncoding 
RNA and mRNA sequence libraries. Reads for all other RNA species 
(tRNA, rRNA, snoRNA, mRNA) are also provided. miRge was designed 
to optimally identify miRNA isomiRs and employs an entropy-based
statistical measurement to identify the differential production of 
isomiRs. It was capable of simultaneously analyzing 100 small 
RNA-Seq samples in 52 minutes, providing integrated analysis
of miRNA expression across all samples.

Oasis is a web application that allows for the fast and flexible online analysis of small RNA-seq (sRNA-seq) data. It was designed
for the end user in the lab, providing an easy-to-use web frontend
including video tutorials, demo data and best practice step-by-step
guidelines on how to analyze sRNA-seq data. Oasis' exclusive selling points are a differential expression module that allows
for the multivariate analysis of samples, a classification
module for robust biomarker detection and an advanced programming
interface that supports the batch submission of jobs. Both modules include the analysis of novel miRNAs, miRNA 
targets and functional analyses including GO and pathway enrichment. Oasis generates downloadable interactive web
reports for easy visualization, exploration, and analysis of data on a local system. Finally, Oasis' modular workflow enables for the rapid (re-)analysis of data.

Small RNA-seq pipeline exceRpt can be used for processing and analyzing the results of the experimental data. Alignment to exogenous 
genomes and their quantification results were used for the analyses
of small RNAs of exogenous origin.

Some single cell tools for normalization and batch correction are also useful. Methods used to quantify mRNA abundance introduce systematic sources of variation that can obscure signals of interest.
Consequently, an essential first step in most mRNA-expression 
analyses are normalization, whereby systematic variations are 
adjusted to make expression counts comparable across genes and/
or samples. Within-sample normalization methods adjust for gene-specific features, such as GC content and gene length, to facilitate 
comparisons of a gene's expression within an individual sample;
whereas between-sample normalization methods adjust for sample-specific features, such as sequencing depth, to allow for comparisons of a
gene's expression across samples. SCnorm is a method for 
between-sample normalization which also allows gene-specific 
features to be adjusted. A number of methods are available for between-sample normalization in bulk RNA-seq experiments. Most of these methods calculate global scale factors (one factor is applied to each sample, and this same factor is applied to all genes in the sample) to adjust for sequencing depth. These methods demonstrate excellent performance in bulk RNA-seq, but they are
compromised in the single-cell setting because of an abundance of zero-expression values and increased technical variability. 
scRNA-seq data show systematic variation in the relationship between transcript-specific expression and sequencing depth 
(which we refer to as the count– depth relationship) that is
not accommodated by a single scale factor common to all genes
in a cell. Global scale factors adjust for a count–depth
relationship that is assumed to be common across genes. 
When this relationship is not common across genes, normalization
via global scale factors leads to overcorrection for weakly and moderately expressed genes and, in some cases, under normalization of highly expressed genes. SCnorm uses quantile regression to estimate the dependence of transcript expression on
sequencing depth for every gene. Genes with similar dependence are then grouped, and a second quantile regression is used to estimate
scale factors within each group. Within-group adjustment for sequencing depth is then performed using the estimated scale factors to provide normalized estimates of expression. Although 
SCnorm does not require experimental RNA spike-ins, performance may
be improved if spike-ins that span the range of expression observed in endogenous genes are available.

scImpute is currently the state-of-art method to do imputation on
scRNA-seq data. It is a statistical method to accurately and robustly
impute the dropouts in scRNA-seq data. scImpute automatically identifies likely dropouts, and only perform imputation on these values without introducing new biases to the rest data. scImpute 
also detects outlier cells and excludes them from imputation. 
Evaluation based on both simulated and real human and mouse 
scRNA-seq data suggests that scImpute is an effective tool to recover transcriptome dynamics masked by dropouts. scImpute is shown to identify likely dropouts, enhance the clustering of cell
subpopulations, improve the accuracy of differential expression analysis, and aid the study of gene expression dynamics. scImpute 
focuses on imputing the missing expression values of dropout genes, 
while retaining the expression levels of genes that are largely
unaffected by dropout events. Hence, scImpute can reduce technical 
variation resulted from scRNA-seq and better represent cell-to-cell
biological variation, while it also avoids introducing excess biases
during its imputation process. To achieve the above goals, scImpute 
first learns each gene's dropout probability in each cell by fitting a 
mixture model for each cell type. Next, scImpute imputes the (highly probable) dropout values of genes in a cell by borrowing information of the same gene in other similar cells, which are selected based on
the genes not severely affected by dropout events. Comprehensive studies on both simulated and real data suggest that compared with the raw scRNA-seq data, the imputed data by scImpute better present cell type identity and lead to more accurate DE analysis results.

\paragraph{conclusion}
In the review, we have discussed the potential application of exRNA 
in diagnosis and prognosis of some complex diseases including cancer.
We have discussed some previous progress and their limitations. 
Some challenges we face when dealing with exRNA data analysis and
cancer prediction. It is clear that an integrative and better tool for exRNA data analysis is essential and useful. We aim to develop
such kind of tool for mapping, expression matrix construction, 
matrix processing, and feature selection. Some data issues include sparsity, fragmentation, heterogeneity and batch effect. We plan to design and apply certain methods to deal with these problems in our project.




\section*{References}


\begin{translationbib}
\item Byron, S. A. et al. Translating RNA sequencing into clinical diagnostics: opportunities and challenges. Nat Rev Genet 17, 257-271, doi:10.1038/nrg.2016.10 (2016).
\item Xi, X. et al. RNA Biomarkers: Frontier of Precision Medicine for Cancer. Non-Coding RNA 3, doi:10.3390/ncrna3010009 (2017).
\item    Ainsztein, A. M. et al. The NIH Extracellular RNA Communication Consortium. J Extracell Vesicles 4, doi:UNSP 27493
10.3402/jev.v4.27493 (2015).
\item De Rubis, G. e. a. Circulating tumor DNA - current state of play and future perspectives. Pharmacol. Res. (2018).
\item    Taylor, J. a. B., M. Proteins regulating microvesicle biogenesis and multidrug resistance in cancer., doi: http://dx.doi.org/10.1002/pmic.201800165 (2018).
\item    van Niel, G. e. a. Shedding light on the cell biology of extracellular vesicles. Nat. Rev. Mol. Cell Biol. (2018).
\item    Torrano, V. e. a. Vesicle-MaNiA: extracellular vesicles in liquid biopsy and cancer. Curr. Opin. Pharmaco (2018).
\item    Melo, S. A. e. a. Glypican-1 identifies cancer exosomes and detects early pancreatic cancer. Nature 523, 177–182 (2015).
\item    Krishnan, S. R. e. a. Isolation of human CD138(+) microparticles from the plasma of patients with multiple myeloma. Neoplasia 18, 25-32 (2016).
\item    Dunham, I. et al. An integrated encyclopedia of DNA elements in the human genome. Nature 489, 57-74, doi:10.1038/nature11247 (2012).
\item    Sparano, J. A. et al. Prospective Validation of a 21-Gene Expression Assay in Breast Cancer. New Engl J Med 373, 2005-2014, doi:10.1056/NEJMoa1510764 (2015).
\item    Vardiman, J. W. et al. The 2008 revision of the World Health Organization (WHO) classification of myeloid neoplasms and acute leukemia: rationale and important changes. Blood 114, 937-951, doi:10.1182/blood-2009-03-209262 (2009).
\item    Garcia-Romero, N. e. a. Extracellularvesiclescompartmentin liquid biopsies: clinical application. Mol. Aspects Med. 60, 27-37 (2018).
\item    Brock, G. et al. Liquid biopsy for cancer screening, patient stratification and monitoring. Transl Cancer Res 4, 280-290, doi:10.3978/j.issn.2218-676X.2015.06.05 (2015).
\item    Tsui, N. B. Y. et al. Maternal Plasma RNA Sequencing for Genome-Wide Transcriptomic Profiling and Identification of Pregnancy-Associated Transcripts. Clin Chem 60, 954-962, doi:10.1373/clinchem.2014.221648 (2014).
\item    Schwarzenbach, H. et al. Clinical relevance of circulating cell-free microRNAs in cancer. Nat Rev Clin Oncol 11, 145-156, doi:10.1038/nrclinonc.2014.5 (2014).
\item    Wozniak, M. B. et al. Circulating MicroRNAs as Non-Invasive Biomarkers for Early Detection of Non-Small-Cell Lung Cancer. Plos One 10, doi:ARTN e0125026
10.1371/journal.pone.0125026 (2015).
\item    Singh, P. K. et al. Serum microRNA expression patterns that predict early treatment failure in prostate cancer patients. Oncotarget 5, 824-840, doi:DOI 10.18632/oncotarget.1776 (2014).
\item    Zhou, J. et al. Plasma MicroRNA Panel to Diagnose Hepatitis B Virus-Related Hepatocellular Carcinoma. J Clin Oncol 29, 4781-4788, doi:10.1200/Jco.2011.38.2697 (2011).
\item    Zaporozhchenko, I. A. e. a. The potential of circulating cell-free RNA as a cancer biomarker: challenges and opportunities. Expert Rev. Mol. Diagn 18, 133-145 (2018).
\item    Valadi, H. et al. Exosome-mediated transfer of mRNAs and microRNAs is a novel mechanism of genetic exchange between cells. Nat Cell Biol 9, 654-U672, doi:10.1038/ncb1596 (2007).
\item    Vickers, K. C. et al. MicroRNAs are transported in plasma and delivered to recipient cells by high-density lipoproteins (vol 13, pg 423, 2011). Nat Cell Biol 17, 104-104, doi:10.1038/ncb3074 (2015).
\item    Nakano, I. et al. Extracellular vesicles in the biology of brain tumour stem cells--Implications for inter-cellular communication, therapy and biomarker development. Semin Cell Dev Biol 40, 17-26, doi:10.1016/j.semcdb.2015.02.011 (2015).
\item    Wei, Z. et al. Coding and noncoding landscape of extracellular RNA released by human glioma stem cells. Nat Commun 8, 1145, doi:10.1038/s41467-017-01196-x (2017).
\item    Hao, Y. X. e. a. KRAS and BRAF mutations in serum exosomes from patients with colorectal cancer in a Chinese population. Oncol. Lett. 13, 3608–3616 (2017).
\item    Manda, S. V. e. a. Exosomes as a biomarker platform for detecting epidermal growth factor receptor-positive high-grade gliomas. J. Neurosurg. 128, 1091–1101 (2018).
\item    Aguado, C. e. a. Fusion gene and splice variant analyses in liquid biopsies of lung cancer patients. Transl. Lung Cancer Res 5, 525–531 (2016).
\item    March-Villalba, J. A. e. a. Cell-free circulating plasma hTERT mRNA is a useful marker for prostate cancer diagnosis and is associated with poor prognosis tumor characteristics. Plos One (2012).
\item    Xu, Y. F. e. a. Plasma exosome miR-196a and miR-1246 are potential indicators of localized pancreatic cancer. Oncotarget 8, 77028–77040.
\item    Jin, X. e. a. Evaluation of tumor-derived exosomal miRNA as potential diagnostic biomarkers for early-stage non-small cell lung cancer using next-generation sequencing. Clin. Cancer Res. 23, 5311–5319.
\item    Liu, Q. e. a. Circulating exosomal microRNAs as prognostic biomarkers for non-small-cell lung cancer. Oncotarget 8, 13048–13058.
\item    Li, Q. e. a. Plasma long noncoding RNA protected by exosomes as a potential stable biomarker for gastric cancer. Tumour Biol. 36, 2007–2012 (2015).
\item    Dong, L. e. a. Circulating long RNAs in serum extracellular vesicles: their characterization and potential application as biomarkers for diagnosis of colorectal cancer. Biomarkers Prev. 25, 1158–1166 (2016).
\item    Tan, S. K. e. a. Serum long noncoding RNA HOTAIR as a novel diagnostic and prognostic biomarker in glioblastoma multiforme. . Mol. Cancer (2018).
\item    Zhang, K. e. a. Genome-wide lncRNA microarray profiling identifies novel circulating lncRNAs for detection of gastric cancer. Theranostics 7, 213–227 (2017).
\item    Josh N. Vo, M. C., Yajia Zhang, ..., Dan R. Robinson, Alexey I. Nesvizhskii, Arul M. Chinnaiyan. The Landscape of Circular RNA in Cancer. Cell (2019).
\item    Capel, B., Swain, A., Nicolis, S., Hacker, A., Walter, M., Koopman, P., Goodfellow, P., and Lovell-Badge, R. Circular transcripts of the testis-determining gene Sry in adult mouse testis. Cell (1993).
\item    Hansen, T. B., Jensen, T.I., Clausen, B.H., Bramsen, J.B., Finsen, B., Damgaard, C.K., and Kjems, J. Natural RNA circles function as efficient microRNA sponges. Nature (2013).
\item    Conn, S. J., Pillman, K.A., Toubia, J., Conn, V.M., Salmanidis, M., Phillips, C.A., Roslan, S., Schreiber, A.W., Gregory, P.A., and Goodall, G.J. The RNA binding protein quaking regulates formation of circRNAs. Cell (2015).
\item    Guarnerio, J., Bezzi, M., Jeong, J.C., Paffenholz, S.V., Berry, K., Naldini, M.M., Lo-Coco, F., Tay, Y., Beck, A.H., and Pandolfi, P.P. Oncogenic Role of Fusion-circRNAs Derived from Cancer-Associated Chromosomal Translocations. Cell (2016).
\item    Legnini, I., Di Timoteo, G., Rossi, F., Morlando, M., Briganti, F., Sthandier, O., Fatica, A., Santini, T., Andronache, A., Wade, M., et al. Circ-ZNF609 Is a Circular RNA that Can Be Translated and Functions in Myogenesis. Mol. Cell (2017).
\item    Pamudurti, N. R., Bartok, O., Jens, M., Ashwal-Fluss, R., Stottmeister, C., Ruhe, L., Hanan, M., Wyler, E., Perez-Hernandez, D., Ramberger, E., et al. Translation of CircRNAs. Mol. Cell (2017).
\item    Schwarzenbach, H. et al. Cell-free nucleic acids as biomarkers in cancer patients. Nat Rev Cancer 11, 426-437, doi:10.1038/nrc3066 (2011).
\item    Yang, H. et al. Genomic variant annotation and prioritization with ANNOVAR and wANNOVAR. Nat Protoc 10, 1556-1566, doi:10.1038/nprot.2015.105 (2015).
\item    Chapman, A. R. et al. Single cell transcriptome amplification with MALBAC. Plos One 10, e0120889, doi:10.1371/journal.pone.0120889 (2015).
\item    Shen, S. H. et al. SURVIV for survival analysis of mRNA isoform variation. Nature Communications 7, doi:ARTN 11548
10.1038/ncomms11548 (2016).
\item    Xia, Z. et al. Dynamic analyses of alternative polyadenylation from RNA-seq reveal a 3 '- UTR landscape across seven tumour types. Nature Communications 5, doi:ARTN 5257
10.1038/ncomms6274 (2014).
\item    Wang, Y. M. et al. Systematic characterization of A-to-I RNA editing hotspots in microRNAs across human cancers. Genome Res 27, 1112-1125, doi:10.1101/gr.219741.116 (2017).
\item    Liang, H. The genomic landscape and clinical relevance of A-to-I RNA editing in human cancers. Cancer Res 76, doi:10.1158/1538-7445.Am2016-2661 (2016).
\item    Kaczor-Urbanowicz, K. E., Kim, Y., Li, F., Galeev, T., Kitchen, R.R., Gerstein, M., Koyano, K., Jeong, S.-H., Wang, X., Elashoff, D., et al. Novel approaches for bioinformatic analysis of salivary RNA sequencing data for development. Bioinformatics 34, 1-8 (2018).
\item    Allen, R. M., Zhao, S., Ramirez Solano, M.A., Zhu, W., Michell, D.L., Wang, Y., Shyr, Y., Sethupathy, P., Linton, M.F., Graf, G.A., et al. Bioinformatic analysis of endogenous and exogenous small RNAs on lipoproteins. J Extracell Vesicles 7 (2018).
\item    Vitsios DM, E. A. Chimira: analysis of small RNA sequencing data and microRNA modifications. Bioinformatics.  31, 3365–3367 (2015).
\item    Baras AS, M. C., Myers JR, et al. miRge - a multiplexed method of processing small RNA-seq data to determine microRNA entropy. PLoS One. (2015).
\item    Capece V, G. V. J., Vidal R, et al. Oasis: online analysis of small RNA deep sequencing data. Bioinformatics 31, 2205–2207. (2015).  
\end{translationbib}

