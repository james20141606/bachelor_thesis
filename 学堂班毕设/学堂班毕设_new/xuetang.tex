\documentclass[a4paper,12pt]{article} 
\usepackage{simpleConference}
\usepackage{times}
\usepackage{graphicx}
\usepackage{amssymb}
\usepackage{url,hyperref}
\usepackage{cite}
\usepackage{amsmath}
\usepackage{txfonts}
\usepackage{booktabs}
\usepackage{algorithm}
\usepackage{algorithmic}
\usepackage{tabularx}
\usepackage[round]{natbib}
\usepackage{fontspec}

%\newcommand{\\citep}[1]{\textsuperscript{\textsuperscript{\cite{#1}}}}

\setmainfont{Times New Roman}

\usepackage{xcolor}
\hypersetup{
    colorlinks,
    linkcolor={black!50!black},
    citecolor={black!50!black},
    urlcolor={black!80!black}
}
\newenvironment{sequation}{\begin{equation}\small}{\end{equation}}
\newenvironment{sealign}{\begin{align}\small}{\end{align}}
\usepackage{setspace}
\linespread{1.5}
\usepackage{indentfirst}
\usepackage[toc,page]{appendix}
\usepackage{subfig}
\DeclareMathAlphabet{\pazocal}{OMS}{zplm}{m}{n} %for fancy L
\usepackage{epsfig, float,array,tabu,longtable,}
%\usepackage{hyperref,wrapfig}
\usepackage{enumerate}
\usepackage[at]{easylist}
\usepackage{graphicx,psfrag}
\usepackage{cite}
\usepackage{sectsty}
\usepackage{enumitem}
\usepackage{caption}
\newlist{inparaenum}{enumerate}{2}% allow two levels of nesting in an enumerate-like environment
\setlist[inparaenum]{nosep}% compact spacing for all nesting levels
\setlist[inparaenum,1]{label=\bfseries\arabic*.}% labels for top level
\setlist[inparaenum,2]{label=\arabic{inparaenumi}\emph{\alph*})}% labels for second level
\usepackage{epstopdf}
\usepackage{amsmath,esint, setspace, fancyhdr, amsfonts,  blindtext}
\usepackage[normalem]{ulem}
\usepackage{tikz}
\usepackage{rotating}
\usepackage[americanvoltages,fulldiodes,siunitx]{circuitikz}
\usepackage{stackengine}
\usetikzlibrary{matrix}
%\usepackage{multirow}
\usetikzlibrary{shapes,backgrounds,patterns}
\usetikzlibrary{mindmap,trees,decorations.markings}
\usetikzlibrary{quotes,angles}
\usepackage{verbatim}
\setlength{\textheight}{8in}
\setlength{\textwidth}{6.5in}
\setlength{\headheight}{0in}
\setlength{\headsep}{0.3in}
\setlength{\topmargin}{0in}
\setlength{\oddsidemargin}{0in}
\setlength{\evensidemargin}{0in}
\setlength{\parindent}{.2in}
\usepackage{algorithmic}
\usepackage{algorithm,float}
\usepackage{caption}





\makeatletter
\newenvironment{breakablealgorithm}
  {% \begin{breakablealgorithm}
   \begin{center}
     \refstepcounter{algorithm}% New algorithm
     \hrule height.8pt depth0pt \kern2pt% \@fs@pre for \@fs@ruled 画线
     \renewcommand{\caption}[2][\relax]{% Make a new \caption
       {\raggedright\textbf{\ALG@name~\thealgorithm} ##2\par}%
       \ifx\relax##1\relax % #1 is \relax
         \addcontentsline{loa}{algorithm}{\protect\numberline{\thealgorithm}##2}%
       \else % #1 is not \relax
         \addcontentsline{loa}{algorithm}{\protect\numberline{\thealgorithm}##1}%
       \fi
       \kern2pt\hrule\kern2pt
     }
  }{% \end{breakablealgorithm}
     \kern2pt\hrule\relax% \@fs@post for \@fs@ruled 画线
   \end{center}
  }
%%%%%%%%%

\begin{document}

\title{\emph{exSEEK: A bioinformatics tool for extra-cellular RNA biomarker discovery}}

\author{
Xupeng Chen \\
Advisor: Prof. Zhi Lu\\
MOE Key Laboratory of Bioinformatics, School of Life Sciences, \\Tsinghua University, Beijing 100084, China\\
  \texttt{xp-chen14@mails.tsinghua.edu.cn} \\
}

\maketitle
\thispagestyle{empty}

\begin{abstract}
Heterosis is the improved or increased function of any biological quality in a hybrid offspring. We have studied yet the largest maize SNP dataset for traits prediction. We develop linear and non-linear models which consider relationships between different hybrids as well as other effect. Specially designed model proved to be efficient and robust in prediction maize's traits.
\end{abstract}

\clearpage

%%%%%%%%%%%%%%%%%%%%%%%%%%%%%%%%%%%%%%%%%%%%%%%%%%%%%%%%%%%%%%
%%%%%%%%%%%%%%%%%%%%%%%%%%%%%%%%%%%%%%%%%%%%%%%%%%%%%%%%%%%%%%
%%%%%%%%%%%%%%%%%%%%%%%%%%%%%%%%%%%%%%%%%%%%%%%%%%%%%%%%%%%%%%


% EXECUTIVE SUMMARY %%%%%%%%%%%%%%%%%%%%%%%%%%%%%%%%%%
\section{Introduction}

\subsection{Liquid biopsy}
In molecular diagnosis, liquid biopsy is an attractive field to both researchers and commercial organizations, because of its non-invasive nature. RNA markers have several advantages over DNA and protein biomarkers, including its higher sensitivity, tissue specificity and lower cost for detection\citep{xi2017rna}. exRNAs (extracellular RNA), defined as RNAs that can be secreted outside of the cell, which make up the majority of detectable RNA species in body fluid, is currently under intensive study.

\subsection{exRNA as biomarkers}
Several works suggested that some exRNAs are functionally active, and may play vital roles in communication between cells under both physiological and pathological conditions\citep{nedaeinia2017circulating}. In mammalian cells, intracellular RNAs are secreted into body fluids in various forms: bound by proteins and form RNPs complex, or wrapped in membrane structure like microvesicles or exosomes\citep{nedaeinia2017circulating}. miRNA, piRNA, fragments of mRNA, snRNA, lncRNA, tRNA and Y RNA have been detected in exRNAs\citep{yuan2016plasma}. Most of the published studies on exRNA biomarker discovery focus on miRNA, as cancers are often accompanied by miRNA dysregulation, and miRNAs are relatively stable in body fluid\citep{mitchell2008circulating}. Extracellular miRNAs are reported to be capable of predicting varies cancers\citep{nedaeinia2017circulating}, \citep{wozniak2015circulating}, \citep{singh2014serum}, \citep{schwarzenbach2014clinical}, \citep{zhou2011plasma}. Recently, the clinical value of other RNA types has also been recognized\citep{li2019extracellular}. For example, several lncRNAs in exRNA were identified as biomarkers for cancers. 
Deep sequencing makes it possible to monitor RNA fragments in the sample simultaneously without prior knowledge of their sequences. exRNA sequencing is becoming a powerful tool for novel biomarker discovery, although many technical problems in both experiment and data processing remain to be solved. RNA concentration in body fluid is much lower than that in the tissue sample, lead to pervasive dropouts in exRNA sequencing data\citep{laurent2015meeting}. Exogenous RNAs are often present in exRNAs, which may interfere assignment of the reads\citep{kaczor2017novel}, \citep{beatty2014small}, \citep{allen2018bioinformatic}. The short length of exRNA also increase the difficulty of correct mapping, and many small RNA sequencing analyzing tools have to alleviate such problem via mapping the reads to different types of RNA in a sequential manner\citep{wu2017srnanalyzer}, \citep{baras2015mirge}. Different exRNA isolation protocols are biased toward certain types of exRNA cargos\citep{srinivasan2019small}, \citep{murillo2019exrna}. Considerable heterogeneity and batch effect can often be observed in exRNA sequencing data, as standard exRNA sequencing protocol still have not been established\citep{laurent2015meeting}, \citep{li2015comparison}. 
exRNAs are highly fragmented. Most of exRNAs, except for miRNA and piRNA, generally exist as small fragments, instead of annotated full-length sequences\citep{wei2017coding}. Previous studies suggested some of these small RNA fragments are not residues of random degradation. A fraction of small RNA fragments, with their counts significantly higher than the background\citep{allen2018bioinformatic}, \citep{wei2017coding}, can be mapped to certain secondary structure, or certain regions in longer transcripts. It has been suggested that some small RNA fragments are derived from longer noncoding RNAs like tRNA, YRNA, vault RNA, circular RNA, snRNA, and snoRNA via specific molecular machinery, and may carry out functions distinct from their parents. Alternatively, only fragments with certain secondary structures or protein binding sites could survive from RNAase degradation. If some of the full length annotated RNAs are adopted as biomarkers, more clinical applicable methods other than deep sequencing, like PCR, may lead to the inconsistent outcome. Compared to full-length tRNA, rRNA, and snRNA, fragments derived from these RNAs perform better in distinguishing exRNA from different body fluid or different subcomponents of exRNAs in same body fluid. Hence, directly utilize recurring small RNA fragments as biomarker candidates seem to be a better choice.

\subsection{Previous Study}
Several bioinformatics tools have been developed for the analysis of small RNA sequencing data. Chimira, Oasis(\citep{capece2015oasis}), miRge and sRNAnalyzer are general software for small RNA sequencing data analysis, without considering the unique properties of exRNA sequencing. ExceRpt\citep{rozowsky2019excerpt} and TIGER were designed for exRNA sequencing data analysis, but these two tools mainly concern the quantification of RNAs, identification of their origins, and downstream analysis like differential expression, instead of biomarker discovery. 

\subsection{Aim of the study}
For liquid biopsy biomarker discovery, beyond proper assignment of the reads, researchers have to go further, otherwise, biological variations that would be sufficient to distinguish cancer from health may be overwhelmed by technical variations such as the difference in sequencing depth and the existence of batch effect\citep{leek2010tackling}. Here we developed exSEEK, a pipeline for identifying potential biomarkers from either short of long exRNA-seq data for several cancers’ classification. After the sequential mapping of the reads, we proposed a new peak calling method to explicitly find domain features of long RNAs. exSEEK can be customized to explicitly preprocess raw expression data with several methods for normalization and removing batch effect; and potential biomarkers can be robustly selected from the combination of recurring RNA fragments from long RNAs and full-length miRNA, via customizable feature selection algorithms. Our novel peak calling method for identifying structured or RBP associated domain in long RNA as features enable us to find domain features of some long RNAs including circular RNA, snoRNA, snRNA, srpRNA, etc.
  


\section{Methods and Results}


\subsection{Data issues and study design}



\begin{figure}[H] % use float package if you want it here
  \centering
  \includegraphics[width = 0.8\textwidth]{data_issues}
  \caption{Issues of exRNA data and strategies used in exSEEK method}
  \label{fig:issues}
\end{figure}

exSEEK solves four major computational analysis issues in exRNA-seq data. As shown in Fig. \ref{fig:issues}. Fragmentation: exRNAs are mostly present as fragments instead of full-length transcripts due to degradation and protection mechanisms. Sparseness: RNA species with low abundance in exRNA-seq datasets cannot be detected due to the limited amount of biofluid samples. Heterogeneity: the abundance of RNAs varies among exRNA-seq datasets obtained from individuals of the same biological group (e.g. healthy donors). Batch effect: the abundance of RNAs varies with factors, e.g. sampling time, library preparation protocol, which is unrelated to the biological group of interest. Using a properly normalized expression matrix, exSEEK discovers biomarkers that best classifies exRNA-seq datasets through robust feature selection based on machine learning framework.


\begin{figure}[H] % use float package if you want it here
  \centering
  \includegraphics[width = 0.8\textwidth]{pipeline}
  \caption{exSEEK frame work}
  \label{fig:frame}
\end{figure}
We can have an overview of the exSEEK framework in Fig. \ref{fig:frame}. It includes three modules, reads process, matrix processing and feature selection. After careful reads mapping, domain calling. We can have the combined features as the expression matrix. Next, we can apply matrix processing methods to normalize read depth and correct batch effect of the expression matrix. We use some metrics to quantify the processing effect. Next we can apply some machine learning models to do robust feature selection and select candidates as biomarkers.


\subsection{exRNA-seq data collection and preprocessing}

\subsubsection{Processing extracellular small RNA datasets}
We downloaded exosomal small RNA-seq data (S-exoRNA-seq) from GEO (accession: GSE71008). GSE71008 contains samples from three cancer types: CRC (100 samples), PRAD (36 samples) and PAAD (6 samples) and 50 samples from healthy donors. We excluded PAAD from our analysis due to the small sample size. The cell-free small RNA-seq dataset was compiled from 4 publications (GSE123972, GSE113994, GSE53080, and GSE94582). The GSE123973 (S-cfRNA-seq) dataset was generated from our previous study and is currently not public in GEO, which contains 30 blood plasma samples from HCC patients and 13 samples from healthy donors. GSE113994, GSE53080, and GSE94582 were all derived from healthy donors. GSE113994 originally contained 216 samples from healthy donors. To reduce sample imbalance between data sources, we filtered samples with less than 2,000,000 reads, > 20\% UniVec-mapped reads (described below) or > 20\% unmapped reads, resulting in 53 samples. The cfRNA dataset containing 132 samples was analyzed using a unified processing pipeline.
In the reads cleaning step, we first trimmed 3' adapters for each read using cutadapt. Reads with an average quality score below 30 or shorter than 16nt after adapter trimming were discarded. In the reads mapping step, we adopted a sequential mapping strategy to map reads to multiple sequence databases to increase sensitivity for detection of RNA types of interest (e.g. noncoding RNA). Bowtie2 was used with the 
“—sensitive'' option and disabled local alignment. Reads were first mapped to the spike-in sequence if spike-in RNA added to the library. The remaining reads were then mapped to UniVec database (from NCBI UniVec database) and rRNA database (from NCBI RefSeq database). Reads mapped to spike-in sequences, UniVec or rRNA were removed.

Remaining reads after the first three steps were then mapped to RNA types of interest according to a pre-defined order: lncRNA, miRNA, mRNA, piRNA, snoRNA, snRNA, srpRNA, tRNA, TUCP RNA, and Y RNA. lncRNA was set to the highest priority to increase the sensitivity of detection of lncRNA peaks. miRNA sequences were downloaded from miRBase. Reads that overlap with mature miRNA annotations were assigned to that mature miRNA. piRNA sequences were downloaded from piRNABank. The genomic locations of other RNA types were extracted from the GENCODE V27 annotation file according to the 
“gene\_type'' attribute in each record. Sequences of all exons that belong to a transcript were joined such that reads could be mapped to splicing junctions without a junction-aware mapping software. For long RNAs with multiple isoforms, we only extracted sequences of the longest isoform to avoid ambiguity for mapping reads to exons shared by multiple isoforms. Reads that could not be mapped to predefined RNA types were mapped to genome sequences (UCSC genome build hg38) and assigned RNA type “other''. Reads of “other'' type were then annotated as “promoter", “enhancer'', “repeats'', “genomic''. Promoter and enhancer regions were defined based on chromatin states discovered by the ChromHMM software from 9 cell lines. ChromHMM tracks were downloaded from the UCSC Genome Browser and converted to hg38 coordinates using CrossMap. Promoters were defined by three ChromHMM states “Active Promoter'', “Weak Promoter'' and “Posed Promoter''. Enhancers were defined by three ChromHMM states “Strong Enhancer", “Weak Enhancer". Repeats regions were downloaded from the RepeatMasker track from UCSC Genome Browser (the “rmsk'' table). Reads that could be mapped to the genome but could not be mapped to promoter, enhancer or repeats were defined as “genomic". In the last step, reads that could not be mapped to the human genome were mapped to circular RNAs from circBase. We only consider reads that span across back-splicing junctions as circular RNA reads. Reads that could not be mapped to circular RNA were defined as “unmapped".


\subsubsection{Processing extracellular long RNA datasets}
Exosomal long RNA-seq (L-exoRNA-seq) datasets were downloaded from exoRBase. 12 CRC, 14 HCC, 14 PAAD, and 32 normal samples were used in this study. Any adapters present in paired-end reads were removed using cutadapt. Read pairs with an average quality score below 30 in either read of the pair were removed. Read pairs were then mapped three sequences databases (rRNA, human genome, circular RNA) using the sequential mapping strategy similar to the small RNA reads mapping strategy. In the genome mapping step, a genome index was generated using STAR with splicing junction annotations from GENCODE V27. Circular RNA mapping was similar to the strategy used in the small RNA mapping pipeline, with the additional requirement that the genomic interval defined by a read pair (from 5’-end in read 1 to 3’-end in read 2) should overlap the back-splicing junction. Only read pairs mapped to the human genome or circular RNA were included in further analysis. We removed duplicated read pairs using picard MarkDuplicates because the dataset contains a large proportion of duplicated reads due to PCR amplification in library construction. Finally, a gene count matrix was generated using featureCounts with options “-t exon -g gene\_id -M -s 1".

\paragraph{Data summary}
The datasets we use in the study are shown in table ~\ref{tab:exRNAsumtable}

\begin{table}[htb]
  \centering
  \begin{minipage}[t]{0.83\linewidth} 
  \caption[exRNA data summary]{exRNA Data summary}
  \label{tab:exRNAsumtable}
  %210pt in total
    \begin{tabularx}{\linewidth}{p{90pt} p{80pt} p{110pt} p{70pt}}
      \toprule[1.5pt]
      { Data type} & {Sources}  & {Sample class} & {Sample size}\\\midrule[1pt]   
      S-exoRNA-seq & GSE71008 & \textbf{CRC}, PRAD, HD & 186\\
      S-cfRNA-seq & GSE123972 & \textbf{HCC}, HD & 43\\
      S-cfRNA-seq & GSE113994 & HD & 53\\
      S-cfRNA-seq & GSE53080 & HD & 17\\
      S-cfRNA-seq & GSE94582 & HD & 20\\
      L-exoRNA-seq & exoRBase & \textbf{HCC}, CRC, PRAD, HD & 20\\
      % (\cite{yuan2016plasma}) (\cite{tan2019noncoding}) (\cite{max2018human}) (\cite{giraldez2018comprehensive}) (\cite{akat2014comparative}) (\cite{li2017exorbase})
      \bottomrule[1.5pt]
    \end{tabularx}
  \end{minipage}
\end{table}

\subsection{exRNA fragments of long RNAs can be recurrently detected in plasma}

\subsubsection{exRNA fragments detection}
The first issue of exRNA data is that RNAs, especially long RNAs, are intrinsically unstable and fragmented in plasma. However, different from RNAs in tissue, we observe that exRNAs are not randomly fragmented in plasma based on both S-cfRNA-seq and S-exoRNA-seq data. We observed the non-uniform distribution of reads along with a most long transcript and found regions with read coverage substantially higher than background (Fig. ~\ref{fig:domain_example}). Because some of the hotspot regions may be due to random variations within a sample and are not suitable to serve as a biomarker, we analyzed the recurrence of such regions among samples. After aligning read coverage of multiple samples at the same transcript, we identified hotspot regions supported by multiple samples. Next, we developed a peak calling algorithm to detect regions with significantly higher reads coverage than background and can be detected in at least 10\% samples. According to the origin of the fragments, the most diverse annotated RNA types was lncRNA. We also detected a large number of fragments that were mapped to the human genome but were not annotated to any known RNA types. Compared to abundant RNA types like miRNA and lncRNA, exRNA domains in other RNA types and unannotated genomic regions were of higher diversity and could potentially include biomarkers that distinguish cancer and normal samples with higher accuracy. 



\begin{figure}[t]
\centering
\subfloat[fragments detected by exSEEK]{%
  \includegraphics[width=8cm,keepaspectratio]{peak_calling}
}\hfill
\subfloat[ Domain of cfRNA and tissue RNA]{%
  \includegraphics[width=8cm,keepaspectratio]{domain_example}
}%
\caption{Domain example in cfRNA and tissue RNA}
    \label{fig:domain_example}
\end{figure}

\subsubsection{Domain detection}
  Domain detection is similar to the peak calling in CLIP-seq/ChIP-seq datasets. However, peak calling for exRNA data is different from CLIP-seq/ChIP-seq datasets in several aspects: (1) a ChIP-seq experiment typically includes an input DNA control library to normalize non-uniform coverage in DNA sequencing, which was not available in most exRNA datasets; (2) CLIP-seq peak calling tools such as PARalyzer, CIMS utilize characteristic events caused by cross-linking to determine the RBP binding sites at single nucleotide resolution, which is not present in exRNA datasets; (3) Universal peak calling tools, e.g. Piranha, detect significant peaks using coverage of the start location of each mapped reads, which may deviate from the true peak location. We developed a domain detection algorithm for exRNA data with similar ideas to Piranha, but could detect domain locations with higher sensitivity and better resolution. 


\subsubsection{ZTNB test-based peak calling}
  We first divided the transcriptome and genome into 20nt bins and count number of reads that overlap each bin with at least 1 nucleotide for each sample. Similar to Piranha, we fitted a zero-truncated negative binomial (ZTNB) model to read counts of all bins with at least one read. P-values were calculated for each bin under the null hypothesis that read counts of the bin is smaller than the mean. Multiple hypothesis correction was applied to the p-values following the BH procedure and bins with adjusted p-values below 0.05 were considered as significant. However, the ZTNB test described above could miss some peaks because the read counts used to fit the null distribution contains bins with significantly large read counts. To increase the sensitivity of domain detection, we removed significant bins identified in the first step of ZTNB test and used the remaining bins for the second step of ZTNB test. The second step of ZTNB test was similar to the first step. The significant bins identified in the two rounds of ZTNB test was then merged into one set. Adjacent bins (distance = 0) were merged into a single domain. 

\subsubsection{Recurrent exRNA fragments are associated with the structural domain and/or protein binding}

To confirm that fragmentation was more characteristic in exRNA dataset than in tissue, we analyzed small RNA-seq data from TCGA HCC samples using the same pipeline. We counted reads of both cfRNA samples and TCGA tissue samples in domains detected using cfRNA samples. We proposed the ratio between read counts in exRNA domain and the original full-length transcript as the measure of fragmentation. If most reads are distributed in a small region within a transcript, the domain/full-length ratio will be close to 1, indicating a high fragmentation level. Otherwise, the number of reads in a small region is expected to be proportional to the length ratio between the region and full-length transcript. We calculated the domain/full-length ratio per transcript averaged across samples in the sample group and found that the ratio in cfRNA samples was significantly higher than in TCGA tissue samples. The conclusion was further supported by an exosomal small RNA-seq dataset and TCGA CRC small RNA-seq dataset.

Small RNA fragments were reported to be generated through small RNA processing mechanism in tissue by previous studies. However, unlike RNAs in tissue, RNAs in plasma degrades during circulation and sample extraction. RNAs that are detected in plasma may be stabilized through various mechanisms including forming stable RNA secondary structure or associating with proteins. To explore the possibility that exRNA domains detected by our method were stabilized through RNA secondary structure, we calculated average structure profiles inside exRNA domains using icSHAPE data determined in HEK293 cell line in vivo. We generated background structure profiles by shuffling locations of exRNA domain among all detected transcripts in the cfRNA dataset. We found that exRNA domains are significantly more structured than the background. We also found that the fraction of domains bound by at least RNA-binding protein was higher than the background. The results are shown in Fig. \ref{fig:ic_rbp}





\begin{figure}[t]
\centering
\subfloat[Structure characteristics of domain and background region revealed by icSHAPE reactivity]{%
  \includegraphics[width=7cm,keepaspectratio]{icshape}
}\hfill
\subfloat[ Domain and background region’s potential to bind with RNA Binding Proteins]{%
  \includegraphics[width=8cm,keepaspectratio]{rbp}
}%
\caption{Structure and RBP binding analysis of domain}
\label{fig:ic_rbp}
\end{figure}

\subsection{Correct data heterogeneity for exRNA}
We generated an expanded count matrix by combine read counts mapped to miRNA and exRNA domains. However, the mapped read counts are affected by RNA extraction efficiency, library size and technical batch effects in additional underlying RNA concentrations. We design the following pipeline (\label{fig:mx_process_pipe}) to do normalization and remove batch effect in the expression matrix.

\begin{figure}[H] % use float package if you want it here
    \centering
    \includegraphics[width = 0.7\textwidth]{matrix_processing}
    \caption{Matrix Processing Pipeline}
    \label{fig:mx_process_pipe}
\end{figure}


A common practice is to calculate counts per million (CPM) to normalize library size, which can also be considered as applying a global normalization factor to each sample. Global normalization methods normalize all genes in the sample by a common size factor which represents the effective library size. Various methods have been developed to calculate the common size factor, including counts per million (CPM), In an RNA composition, all gene expression values are affected by the variation of a single gene, which is more obvious when the RNA composition is dominated by a few genes. The effect of abundant genes on gene expression values can be corrected by more robust methods for size factor calculation: relative log expression (RLE), trimmed mean of M-values (TMM). 
We demonstrated the effect of different normalization methods using RLE (relative log expression) plots (Fig. \ref{fig:rle}). Relative log expression of a gene is log expression of a gene subtracted by median log expression of that gene across samples. Based on the assumption that most genes are not differentially expressed between samples, we expect that properly normalized expression of all genes should be distributed around zero. We observed that after normalization, the dispersion of relative log expression was reduced by RLE.

\begin{figure}[H] % use float package if you want it here
    \centering
    \includegraphics[width = 0.8\textwidth]{RLE}
    \caption{Effect of library size normalization}
    \label{fig:rle}
\end{figure}

We applied three methods that are commonly used to correct batch effects for microarray or RNA-seq data: ComBat, RUVs and limma. ComBat and limma require known batch information, while RUVs discovers unknown batch effects from data. The major batch effect in the cfRNA dataset was their original publication, including library construction protocol sample extraction, etc. The dataset GSE94582 was divided into three batches because they were generated by different library construction protocols. We provided the batch effect as a categorical variable for limma and ComBat, and no batch information for RUVs.
To demonstrate the results of batch effect correction, we plot samples in GSE123972, which were expected to be mixed in the same cluster. We performed variance decomposition analysis on the expression matrix, which measures batch effect by linear regressing expression values of each gene by class label and the batch variable (Fig. \ref{fig:batch_correction}). After batch effect correction, the variance explained by the batch variable.
\begin{figure}[H] % use float package if you want it here
    \centering
    \includegraphics[width = 0.65\textwidth]{batch_correction}
    \caption{batch correction evaluation}
    \label{fig:batch_correction}
\end{figure}

To evaluate the effect of normalization by combinations of library size normalization methods and batch effect removal methods, we calculated two metrics for each corrected expression matrix: unsupervised clustering accuracy (UCA) and mkNN score ~\ref{eq:mknn}. UCA measures the consistency of sample labels by unsupervised clustering and true class labels, which reflects how biological variation is preserved after correction. mkNN measures the extent of samples from different batches are mixed in the same class. We expected that a perfectly corrected expression matrix should be assigned a UCA score of 1 and mkNN score of 1. 
To demonstrate the results of matrix processing, we plot samples by the first two principal components after PCA (Fig. \ref{matrix_processing_metric}). In the expression matrix normalized by CPM, normal samples from GSE94582, GSE113994 and GSE53080 were well separated from normal samples in GSE123972, which were expected to be mixed in the same cluster. After batch effect correction, normal samples from all publications became closer, while remained separated from HCC samples.
By comparing the two metrics of all combinations of methods, we recommend that CPM should be combined with ComBat or limma for the cfRNA dataset.



\begin{equation}
\label{eq:mknn}
\frac { 1 } { B } \sum _ { b = 1 } ^ { B } \frac { \overline { x } _ { b } - k N _ { b } / ( N - 1 ) } { \min \left( k , N _ { b } \right) - k N _ { b } / ( N - 1 ) }
\end{equation}

\begin{figure}[H] % use float package if you want it here
  \centering
  \includegraphics[width = 0.8\textwidth]{matrix_processing_metric}
  \caption{Use UCA and mKNN score to evaluate matrix processing effect}
  \label{fig:matrix_processing_metric}
\end{figure}



\subsection{Expression of exRNA domains predicts cancer better than full-length transcripts}

The expression matrix processed by the matrix processing flow eliminates sample library size inconsistency and batch effect problems caused by some technical variance. We can apply some statistical learning models to the processed expression matrices, design a set of feature selection processes, and select good, stable, biologically meaningful features for cancer and normal sample classification as potential biomarkers for cancer detection. Things.


\subsubsection{Differential expression anlysis}
Differential expression analysis use statistical models to find genes that are differentially expressed between cancer and normal people on a gene-by-gene basis. It can also be used as one of the basic methods of feature selection.

We use the DESeq2 package for differential expression analysis. It is worth noting that DESeq2 requires that the input matrix must be a gene count matrix, without normalizing the counts (\cite{love2014moderated}). Because such a matrix has the most resourceful information, DESeq2 automatically performs the normalization by itself.

To compare counts between two groups. DESeq2 builds a model for the observed counts. This model has some parameters: (1) a normalization parameter, for differences in library size at least, or it can be extended by other software; (2) a variance parameter, called dispersion; (3) parameters representing the group differences. Fit (1) using the same method from the original DESeq. Fit (2) in two steps: first find the value of the parameter that makes the  likelihood largest, which is called maximum likelihood estimation. Look at all the values from all of the genes and move these values towards a middle value. Bayes theorem guides the amount of movement for each gene: if the information for the gene is low, the value is moved more to the middle, if the information for the gene is high, the value is moved very little. Fit (3) using the same technique as used for (2). The values for (3) are a useful final product, as are sets of genes where the group differences are likely to be above a threshold (zero or otherwise). These sets are defined by their false discovery rate (FDR).

For the selection of differentially expressed genes, here we use an improved indicator to replace FDR, which is shown in the equation~\ref{equ:demetric}. The metric considers FDR and fold change as a whole to select better candidates.

\begin{equation}\label{equ:demetric}
  \pi = \Vert \log _ { 2 } F C \Vert \cdot \left( - \log _ { 10 } \mathrm { FDR } \right)
\end{equation}


Using the DESeq2 to analyze the original gene count matrix, we can get the results of the graph ~\ref{fig:lulab_de} and the graph ~\ref{fig:scirep_de}.


For ~\ref{fig:lulab_de}, we used S-cfRNA-seq data to detect differentially expressed genes in liver cancer and its early stage. Comparing them with normal samples, respectively, we can find out the differential genes that can distinguish between normal samples and liver cancer as well as normal samples and early liver cancer. (A) A volcano plot showing the log-fold change and log-adjusted p-value of each gene. The red dots in the figure are differentially expressed genes, and the closer to the upper right and upper left genes, the more obvious the differential expression; (B) The fold change, expression value and log-adjusted p-value of the top ten differentially expressed genes; (C) Heatmap of the differentially expressed genes were selected to show their classification effects on cancer and normal samples.

\begin{figure}[H] % use float package if you want it here
    \centering
    \includegraphics[width = 0.8\textwidth]{lulab_de}
    \caption{S-cfRNA-seq data differential expression analysis}
    \label{fig:lulab_de}
\end{figure}

For ~\ref{fig:scirep_de}, we used S-exoRNA-seq data to detect differentially expressed genes in colorectal cancer and its early stage. Comparing them with normal samples, respectively, we can find out the differential genes that can distinguish between normal samples and colorectal cancer as well as normal samples and early liver cancer. (A) A volcano plot showing the log-fold change and log-adjusted p-value of each gene. The red dots in the figure are differentially expressed genes, and the closer to the upper right and upper left genes, the more obvious the differential expression; (B) The fold change, expression value and log-adjusted p-value of the top ten differentially expressed genes; (C) Heatmap of the differentially expressed genes were selected to show their classification effects on cancer and normal samples.
\begin{figure}[H] % use float package if you want it here
    \centering
    \includegraphics[width = 0.8\textwidth]{scirep_de}
    \caption{S-exoRNA-seq differential expression analysis}
    \label{fig:scirep_de}
\end{figure}

Models for differential expression analysis generally consider the contribution of each feature (ie, gene) independently, and cannot consider the joint contribution of features to classification. Next we will use some classic machine learning models to better combine different features to select a set of biologically more explanatory and representative genes as possible biomarkers. At the same time, considering the generalization ability and stability that the model must have, we will also design a framework of feature selection to better select features.


\subsubsection{feature selection algorithm}

\begin{algorithm}
  \caption{Robust feature selection algorithm}\label{alg:fs_algorithm}
  \begin{algorithmic}[1]
  \STATE Scale each feature independently using robust normalization;
  \FOR{feature\_num $k \in [1,10]$}
  \STATE Random select 90\% samples for 50 times;
  \STATE Run a classifier (Random Forest, Logistic Regression or Linear SVM) to select features based on feature importance. 
    \FOR{each classifier $\in$ [Random Forest, Logistic Regression or Linear SVM]}
    \STATE Optimize classifier’s hyper-parameters by 5-fold cross-validation;
    \ENDFOR
  \STATE Select top k features that are recurrently selected across resampling runs;
  \STATE Calculate AUC mean;
  \ENDFOR
  \STATE Rank processing method by AUC, select the best $k$;
  %\STATE Select union of features in different feature\_num setting. Refit by selecting $I \in [1,10]$ features and calculate feature importance.
  \end{algorithmic}
  \end{algorithm}


We developed a robust framework for classification and feature selection. We use the framework for resampling of samples and cross-validation to select features as shown in ~\ref{alg:fs_algorithm}. At first, we apply a robust feature scaling using (3rd quantile – 1st quantile) as a scaling factor. For the best preprocessing method selected in the matrix processing part, we test feature numbers including 1\~10, 20, 30, 40, and 50. For each feature number, we would like to retain, we will do resampling of samples 50 times, retaining 80\% samples randomly each time. We test three models including random forest, logistic regression, and linear SVM. In each time, we do a 3-fold cross validation for model hyperparameters optimization and retain features with the highest weight in the model. After 50 times resampling process, select top features recurrently selected across resampling runs as robust features. We also calculate Area under the ROC Curve (AUC) to quantify classification performance in each run and calculate its mean in 50 times resampling runs. At last, we select the union of features in different$feature_num$ setting. Refit by selecting 1\~10 features and calculate feature importance.

There are two metrics we use to evaluate our framework’s classification performance and select the best classification model and best feature type. We use AUC to quantify classification performance and Kuncheva index (equation~\ref{eq:ki})to measure the stability of a feature selection method. We apply a comprehensive feature selection strategy evaluation using GSE71008 CRC data and GSE123972 HCC data in (Fig. 4a, 4b). It shows that robust logistic regression is better than random forest in both AUC and stability and domain feature better predicts cancer samples than features with the only miRNA which are widely used as biomarkers in previous studies. We also compare our robust feature selection framework with simple feature selection only using feature importance of a classification model to select a fixed number of features without resampling (depicted as robust and max in the plot). It shows that a robust feature selection framework (robust) performs better than simple feature selection (max) in both AUC and stability.

\begin{equation}\label{eq:ki}
\mathrm { KI } \left( f _ { I } , f _ { j } \right) = \frac { \left| f _ { I } \cap f _ { j } \right| N - d ^ { 2 } }{ d N - d ^ { 2 } } = \frac { \left| f _ { I } \cap f _ { j } \right| - d ^ { 2 } / N } { d - d ^ { 2 } / N }
\end{equation}

Applying the robust feature selection to three datasets GSE123972, GSE71008, and exoRBase, we could have the average AUC of four cancers in different datasets and especially the early stage classification result of HCC and CRC in GSE123972 and GSE71008 respectively in (Fig. \ref{fig:auc_bar_fig4c_5}).

\begin{figure}[H] % use float package if you want it here
  \centering
  \includegraphics[width = 0.7\textwidth]{auc_bar_fig4c_5}
  \caption{Four cancers classification AUC in three datasets}
  \label{fig:auc_bar_fig4c_5}
\end{figure}

\subsubsection{Classification evaluation}
To evaluate some certain biomarkers and candidates selected by exSEEK, we have AUC bar plot of these markers of HCC and CRC in GSE123972 and GSE71008 shown in Fig. \ref{fig:hcc_crc_auc_sum}. miR-20 and miRNA6, miRNA6 (use exSEEK to refit), ncRNA10 (selected by exSEEK) were tested on GSE71008, lRNA10 (selected by exSEEK) were tested on our own dataset (PICO). CRC/HD means a binary classification of CRC and healthy donor, S1/HD means a binary classification of CRC stage1 and healthy donor. The numbers in bracket indicated feature number. We also have AUC summary of some known biomarkers and selected candidates for HCC. miR-21 and miRNA7 (use original parameters), miRNA7 (use exSEEK to refit), ncRNA10 (selected by exSEEK) were tested on GSE123972, lRNA10 (selected by exSEEK) were tested on exoRBase. HCC/HD means binary classification of HCC and healthy donor, S1/HD means a binary classification of HCC stage1 and healthy donor. Previously, 7 miRNAs have been used as a panel for liver cancer diagnosis (ref.). We have compared the performance of the published model (ref) and a model re-built by exSEEK using the same 7 miRNAs. The dramatic improvement demonstrates both the robustness of 7 miRNAs revealed previously and the importance of data normalization and processing steps we have engineered in exSEEK.





\begin{figure}[t]
\centering
\subfloat[Biomarkers classification performance in HCC]{%
  \includegraphics[width=8cm,keepaspectratio]{hcc_auc_sum_revise_fig4b_5}
}\hfill
\subfloat[Biomarkers classification performance in CRC]{%
  \includegraphics[width=8cm,keepaspectratio]{crc_auc_sum_revise_fig4b_5}
}%
\caption{AUC summary of HCC and CRC}
\label{fig:hcc_crc_auc_sum}
\end{figure}

To illustrate the classification performance of features selected by our framework. We plot the Receiver Operating Characteristic (ROC) curve of our selected features and previously reported biomarkers in (Fig. \ref{fig:roc_curve}). Our selected features perform better than previous biomarkers in AUC. It has a sensitivity of 0.966, the specificity of 0.909 and precision (PPV) of 0.749 in balanced point. The balanced point was selected using some metrics explained in Matrix processing evaluation part.

\begin{figure}[H] % use float package if you want it here
  \centering
  \includegraphics[width = 0.4\textwidth]{roc_curve}
  \caption{ROC curve of HCC early stage classification result}
  \label{fig:roc_curve}
\end{figure}

We also visualize the heatmap of selected features and its weight, log CPM, fold change and its p-value in differential expression and plot heatmap in (Fig. \ref{fig:candidate_sum}). Weight, recurrence, fold change and -log10(q value) of selected candidates were visualized by scatter and bar plot. 

\begin{figure}[H] % use float package if you want it here
  \centering
  \includegraphics[width = 1\textwidth]{candidate_sum}
  \caption{Analysis of selected features}
  \label{fig:candidate_sum}
\end{figure}

We further compare different feature number in feature selection. It shows we can achieve similar performance using less (3 or 4) features (Fig. \ref{fig:compare_feature_num})
 
 



\begin{figure}[H] % use float package if you want it here
    \centering
    \includegraphics[width = 0.8\textwidth]{compare_feature_num}
    \caption{AUC comparison when selecting different number of candidates}
    \label{fig:compare_feature_num}
\end{figure}




\subsection{exSEEK software usage}

\subsubsection{exSEEK modules}
We developed exSEEK software to do the exRNA analysis and feature selection, we use snakemake (\cite{koster2012snakemake})to control the process. exSEEK contains three main modules shown in Fig. \ref{fig:exseekmodule}. The Utilities and Preprocess modules contain reads preprocessing, mapping and expression matrix construction. exSEEK module can do differential expression, matrix processing, feature selection. 

The users only need to fill in one config files and use very simple commands to run the software. Users should complete the following steps before running the command:


\begin{itemize}
  \item Install exseek and requirements
  \item Prepare genome and annotation
  \item prepare input files in right file path
  \item Set up configuration
\end{itemize}

Then users can run the command, by specifying the module users want to run and dataset users provide.



\begin{figure}[t]
\centering
\subfloat[exSEEK software]{%
  \includegraphics[width=8cm,keepaspectratio]{exSEEK_module}
}\hfill
\subfloat[ exSEEK software commands]{%
  \includegraphics[width=8cm,keepaspectratio]{command}
}%
\caption{exSEEK modules and commands}
    \label{fig:exseekmodule}
\end{figure}


\subsubsection{Plotting modules of exSEEK}


exSEEK also contains highly interactive and standard plotting modules shown in Fig. \ref{fig:jupyter}, we use jupyter-notebook\citep{kluyver2016jupyter} to construct plotting module. Users can obtain high standards plots using interactive settings.

\begin{figure}[H] % use float package if you want it here
    \centering
    \includegraphics[width = 0.8\textwidth]{jupyter}
    \caption{exSEEK plotting modules}
    \label{fig:jupyter}
\end{figure}





%%%%%%%%%%%%%%%%%%%%%%%%%%%%%%%%%%%%%%%%%%%%%%%%%%%%%%%%%%%%%%
%%%%%%%%%%%%%%%%%%%%%%%%%%%%%%%%%%%%%%%%%%%%%%%%%%%%%%%%%%%%%%
%%%%%%%%%%%%%%%%%%%%%%%%%%%%%%%%%%%%%%%%%%%%%%%%%%%%%%%%%%%%%%

\section{Discussion}

In summary, through literature research and preliminary exploration, we have determined the possibility of exRNA as a molecular marker of cancer, and the novelty of non-coding RNA other than miRNA and lncRNA as molecular markers. Our preliminary research on non-coding RNA functions and algorithms, database development, and exRNA library construction and sequencing and analysis pipelines also ensure the feasibility of the project.

As described in the background, many small RNAs (miRNAs) and long RNAs lncRNAs have been used as diagnostic markers for cancer. Functionally, many studies have shown that many kinds of RNA carried by exosome are closely related to tumorigenesis, angiogenesis and tumor metastasis. With the development of sequencing technology, accurate expression of exRNA at the genome-wide level can detect RNA editing sites and editing levels, and improve the accuracy and reliability of using exRNA as a molecular marker. The feasibility of medium-length RNA sequencing technology also proves that long RNA can be used well as a marker for liquid biopsy.
High-quality RNA sequencing data is the basis of downstream analysis, including biomarker discovery. exRNA sequencing is a challenging field, and there is large room for improvement in experimental steps like RNA isolation and library construction. Technical difficulties still exist for precise quantification of exRNA via RNA sequencing. exRNA sequencing data contains a large amount of duplicated reads, which is formerly described as artifacts generated in PCR and cannot reflect the real abundance of certain RNA species. This problem may be relieved by the application of a unique molecular identifier (UMI). The number of samples should be as large as possible, as the small number of samples compared to the high dimensionality of transcripts is a universal problem in the analysis of RNA sequencing data. If the number of samples is too small, it’s impossible for any algorithm to identify biomarkers robustly. The body fluid is usually lack in stably expressed RNA species, some researchers suggested adding spike-in before sequencing can be a proper way for normalization. The experiment should be carefully designed, if the types of the samples overlap with the RNA extraction or library construction batches, the biological variations are very likely to be overwhelmed by the batch effect; even if the algorithm for removing unexpected variations are explicitly applied, the biological variation might be removed together with the confounders.
exSEEK is a preliminary attempt for extracting and selecting biomarkers from exRNA sequencing data. Although exSEEK can be applied for both small RNA and long RNA sequencing data, it was optimized for small RNA. From long RNA sequencing data, we may be able to extract more robust biomarkers from the information of RNA splicing event. Recently, circRNAs in body fluid were suggested to be a reliable biomarker for several cancers, exSEEK haven’t incorporated the information of circRNA into the analyzing pipeline.


\bibliographystyle{plainnat}

\bibliography{refs}
\end{document}
