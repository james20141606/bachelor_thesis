\chapter{Discussion}

In summary, through literature research and preliminary exploration, we have determined the possibility of exRNA as a molecular marker of cancer, and the novelty of non-coding RNA other than miRNA and lncRNA as molecular markers. Our preliminary research on non-coding RNA functions and algorithms, database development, and exRNA library construction and sequencing and analysis pipelines also ensure the feasibility of the project.

As described in the background, many small RNAs (miRNAs) and long RNAs lncRNAs have been used as diagnostic markers for cancer. Functionally, many studies have shown that many kinds of RNA carried by exosome are closely related to tumorigenesis, angiogenesis and tumor metastasis. With the development of sequencing technology, accurate expression of exRNA at the genome-wide level can detect RNA editing sites and editing levels, and improve the accuracy and reliability of using exRNA as a molecular marker. The feasibility of medium-length RNA sequencing technology also proves that long RNA can be used well as a marker for liquid biopsy.
High-quality RNA sequencing data is the basis of downstream analysis, including biomarker discovery. exRNA sequencing is a challenging field, and there is large room for improvement in experimental steps like RNA isolation and library construction. Technical difficulties still exist for precise quantification of exRNA via RNA sequencing. exRNA sequencing data contains a large amount of duplicated reads, which is formerly described as artifacts generated in PCR and cannot reflect the real abundance of certain RNA species. This problem may be relieved by the application of a unique molecular identifier (UMI). The number of samples should be as large as possible, as the small number of samples compared to the high dimensionality of transcripts is a universal problem in the analysis of RNA sequencing data. If the number of samples is too small, it’s impossible for any algorithm to identify biomarkers robustly. The body fluid is usually lack in stably expressed RNA species, some researchers suggested adding spike-in before sequencing can be a proper way for normalization. The experiment should be carefully designed, if the types of the samples overlap with the RNA extraction or library construction batches, the biological variations are very likely to be overwhelmed by the batch effect; even if the algorithm for removing unexpected variations are explicitly applied, the biological variation might be removed together with the confounders.
exSEEK is a preliminary attempt for extracting and selecting biomarkers from exRNA sequencing data. Although exSEEK can be applied for both small RNA and long RNA sequencing data, it was optimized for small RNA. From long RNA sequencing data, we may be able to extract more robust biomarkers from the information of RNA splicing event. Recently, circRNAs in body fluid were suggested to be a reliable biomarker for several cancers, exSEEK haven’t incorporated the information of circRNA into the analyzing pipeline.
