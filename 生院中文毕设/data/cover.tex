\thusetup{
  %******************************
  % 注意:
  %   1. 配置里面不要出现空行
  %   2. 不需要的配置信息可以删除
  %******************************
  %
  %=====
  % 秘级
  %=====
  secretlevel={秘密},
  secretyear={10},
  %
  %=========
  % 中文信息
  %=========
  ctitle={基于细胞外RNA测序数据的生物标志物鉴定工具开发},
  cdegree={理学学士},
  cdepartment={生命科学学院},
  cmajor={生命科学},
  cauthor={陈旭鹏},
  csupervisor={鲁志\ 教授},
 % cassosupervisor={陈文光教授}, % 副指导老师
 %ccosupervisor={某某某教授}, % 联合指导老师
  % 日期自动使用当前时间,若需指定按如下方式修改:
  % cdate={超新星纪元},
  %
  % 博士后专有部分
  cfirstdiscipline={计算机科学与技术},
  cseconddiscipline={系统结构},
  postdoctordate={2009年7月——2011年7月},
  id={编号}, % 可以留空: id={},
  udc={UDC}, % 可以留空
  catalognumber={分类号}, % 可以留空
  %
  %=========
  % 英文信息
  %=========
  etitle={An Introduction to \LaTeX{} Thesis Template of Tsinghua University v\version},
  % 这块比较复杂,需要分情况讨论:
  % 1. 学术型硕士
  %    edegree:必须为Master of Arts或Master of Science(注意大小写)
  %             “哲学、文学、历史学、法学、教育学、艺术学门类,公共管理学科
  %              填写Master of Arts,其它填写Master of Science”
  %    emajor:“获得一级学科授权的学科填写一级学科名称,其它填写二级学科名称”
  % 2. 专业型硕士
  %    edegree:“填写专业学位英文名称全称”
  %    emajor:“工程硕士填写工程领域,其它专业学位不填写此项”
  % 3. 学术型博士
  %    edegree:Doctor of Philosophy(注意大小写)
  %    emajor:“获得一级学科授权的学科填写一级学科名称,其它填写二级学科名称”
  % 4. 专业型博士
  %    edegree:“填写专业学位英文名称全称”
  %    emajor:不填写此项
  edegree={Doctor of Engineering},
  emajor={Computer Science and Technology},
  eauthor={Xue Ruini},
  esupervisor={Professor Zheng Weimin},
  eassosupervisor={Chen Wenguang},
  % 日期自动生成,若需指定按如下方式修改:
  % edate={December, 2005}
  %
  % 关键词用“英文逗号”分割
  %ckeywords={\TeX, \LaTeX, CJK, 模板, 论文},
  %ekeywords={\TeX, \LaTeX, CJK, template, thesis}
}

% 定义中英文摘要和关键字
\begin{cabstract}
体液中的细胞外RNA(exRNA)可以为肿瘤鉴定提供了大量候选生物标记物,
从而实现癌症的早期检测。深度测序使得全面地检测exRNA成为可能。
由于exRNA测序数据的独特性质,鉴定临床使用的生物标志物仍然是一项
非常有挑战性的工作。我们为此开发了exSEEK, 作为一种用于鉴定与癌症
相关的生物标志物的生物信息学工具,可用于不同来源(细胞外游离、外泌体分泌)
以及不同测序方法(小RNA测序、长RNA测序)产生的exRNA数据分析及
生物标志物鉴定。

exRNA测序数据具有高度碎片化、稀疏性、异质性和批次效应等特征。
在这项工作中,我们发现其中一些片段具有反复出现的模式。
与全长转录物相比,重复片段或“结构域特征”在预测癌症方面表现更好,
并且与基于qPCR的测定结果具有更高的一致性。exSEEK可以用于结构域
检测并将其作为特征,对于小RNA测序,将exSEEK可以将exRNA结构域
和miRNA的丰度组合以产生计数矩阵。我们还将多种样本库大小归一化和
去除批次效应方法进行组合,使用非监督聚类准确性(UCA)和m-K最近邻
(mKNN)指标衡量其对计数矩阵的归一化和去除批次效应效果。
我们还开发了一个特征选择框架,使用逻辑斯谛回归,随机森林等机器
学习模型用于选择区分癌症和正常样本的有效且稳定的特征。我们使用
exSEEK对三个数据集进行了综合分析并评估了exSEEK在肝癌,结肠癌
和前列腺癌等癌症的分类性能,取得了良好的效果。

\end{cabstract}

% 如果习惯关键字跟在摘要文字后面,可以用直接命令来设置,如下:
\ckeywords{液体活检, 生物标志物, 特征选择, 机器学习}

\begin{eabstract}
Extracellular RNAs (exRNAs) in body fluid provide a large 
repository of biomarker candidates. Deep sequencing makes it possible to monitor exRNAs in a comprehensive way. 
Because of the unique properties of exRNA sequencing data, 
identification of potential biomarkers for clinical usage 
remains challenging. exSEEK, a bioinformatics tool for 
identification of biomarkers associated with certain 
diseases, which is suitable for analyzing of both small 
and long RNA sequencing data generated by both cell 
free or exosome RNA.

exRNA-seq data are difficult to deal with since it is highly fragmented, sparse heterogeneous and has batch effect. In this work, we showed that some of these fragments have a recurring pattern. Compared to full-length transcripts, the recurring fragments, or "domain features", perform better in predicting cancers, and has higher concordance with the result of qPCR-based assays. 
For small RNA-seq data, exSEEK assigns reads to multiple 
RNA types sequentially in a user-specified order.  One of the unique features of exSEEK is that regions with significantly higher read coverage than background are detected to generate "domain" features. The abundance of exRNA domains and
miRNAs are combined to create a count matrix. We 
applied various combinations of normalization and 
batch removal methods to the count matrix to correct 
data heterogeneity and batch effects. We evaluate 
the normalization and batch correction result using 
unsupervised clustering accuracy (UCA) and m-K-nearest 
neighbor (mKNN). We also developed a feature selection 
framework wrapping some machine learning models 
including logistic regression and random forest to 
robustly select the most important features that distinguish 
cancer from normal samples. We performed integrative 
analysis of three datasets: cell-free small RNA, 
exosomal small RNA and exosomal long RNA and evaluated 
the classification performance of HCC, CRC, PRAD, and PAAD.
\end{eabstract}

\ekeywords{liquid biopsy, biomarker, feature selection, machine learning}
